\renewcommand{\sp}{~~~~~}
\newcommand{\ft}{\phi}
\newcommand{\fts}{\varphi}
\newcommand{\tty}{\rho}
\newcommand{\utty}{\hat{\rho}}
\newcommand{\sembox}[1]{\hfill \normalfont \mbox{\fbox{\(#1\)}}}
\newcommand{\sempart}[2]{\textrm{\textit{#1 \sembox{#2}}}}
\newcommand{\vdb}{\vd_\textrm{b}}
\newcommand{\vde}{\vd_\textrm{e}}
\newcommand{\vdp}{\vd_\textrm{p}}
\newcommand{\vdhp}{\vd_{\hat{\textrm{p}}}}
\newcommand{\vdP}{\vd_\textrm{P}}
\newcommand{\vda}{\vd_\textrm{a}}
\newcommand{\fracc}[2]{\begin{eqnarray} \frac{\begin{array}{c} #1
    \end{array}}{\begin{array}{c} #2 \end{array}} \end{eqnarray}}
\newcommand{\fraccn}[2]{\refstepcounter{equation}\mbox{$\frac{\begin{array}{c} #1 \end{array}}{\begin{array}{c} #2 \end{array}}$}~(\arabic{equation})}
\newcommand{\fraccc}[2]{\mbox{$\frac{\begin{array}{c} #1 \end{array}}{\begin{array}{c} #2 \end{array}}$}}
\newcommand{\fn}{\ensuremath{\lambda}}
\newcommand{\Fn}[3]{\fn#2:~#1~\rightarrow #3}
\newcommand{\FnU}[2]{\fn#1~\rightarrow #2}
\newcommand{\ubar}[1]{\underline{#1}}
\newcommand{\utau}{\hat{\tau}}
\newcommand{\etau}{\ubar{\tau}}
\newcommand{\opty}{\ubar{\phi}}
\newcommand{\Fun}[4]{\mbox{\lstinline[mathescape]!let\ #1\ #2\ :\ #3 =\ #4!}}

\chapter{Overview and Uniqueness Types}

The simplification engine encompasses well established optimisations
such as inlining, copy propagation, constant folding,
common-subexpression elimination, dead-code elimination, and hoisting
of invariant terms out of recurrences in loops and SOACs.  The
simplification rules are critical for optimising the result of
higher-level optimisations, such as producer-consumer
fusion~\cite{henriksen2014exploiting,henriksen2013t2}, hoisting, and
size analysis, which is the subject of
Chapter~\ref{chap:size-analysis}.  For example, size analysis is
implemented via high-level transformation rules that only guarantee
that the asymptotic complexity (in number of operations) of the
original program is preserved, but rely on the simplification engine
to reduce the overhead to be negligible in the common case.

\section{Abstract Syntax}

In this section we describe the abstract syntax of a simplified form
of the Futhark intermediate representation (IR).  Whenever $z$ is an
object of some kind, we write $\bar{z}$ to range over sequences of
objects of this kind. When we want to be explicit about the size of a
sequence $\bar{z} = z_0,\cdots,z_{(n-1)}$, we often write it on the
form $\nseq{z}{n}$ and we write $z,\bar{z}$ to denote the sequence
$z,z_0,\cdots,z_{(n-1)}$.  Depending on context, the elements of the
sequence or may not have separators, or be merely juxtaposed.  For
example, we may use the same notation to shorten a function
application
\[
  f~\nseq{v}{n} \equiv f~v_{1}~\cdots~v_{n}
\]
or a tuple
\[
  (\nseq{v}{n}) \equiv (v_{1}, \ldots, v_{n})
\]
or a function type
\[
  \nseq{\tau}{n} \rightarrow \tau_{n+1} \equiv \tau_{1} \rightarrow \cdots \rightarrow \tau_{n} \rightarrow \tau_{n+1}.
\]
It is always clear from the context which separator (if any) is
intended.  On occasion we use more complicated sequences, where not
all terms under the bar variant.  In this case, the variant term is
subscripted with $i$.  For example,
\[
  (\nseq{[d]v_{i}}{n}) = ([d]v_{1}, \ldots, [d] v_{n})
\]
and
\[
  (\nseq{[d_{i}]v_{i}}{n}) = ([d_{1}]v_{1}, \ldots, [d_{n}] v_{n})
\]

\begin{figure*}
$$
  \begin{array}{lrlr}
    f & ::= & \textbf{id} & \mbox{(Function names)}\\
    d,x,y,z & ::= & \textbf{id} & \mbox{(Variable names)}\\
    c & ::= & \textbf{const} & \mbox{(Constant value)}\\
    t &::=& \texttt{bool}~|~\texttt{i32}~|~\texttt{f32}~|~... & \mbox{(Built-in types)} \\
    \\
    \tau & ::= & \mbox{\texttt{t} ~ $|$ ~ \texttt{[]$\tau$}} & \mbox{(Scalar/array type)} \\
    \utau & ::= & \mbox{$\tau$ ~ $|$ ~ \texttt{*}$\tau$} & \mbox{(Nonunique/Unique type)} \\
    \tty  & ::=   & (\tau_1, \ldots, \tau_n) & \mbox{(Tuple types)}\\
    \utty  & ::=   & (\utau_1, \ldots, \utau_n) & \mbox{(Nonunique/Unique tuple types)}\\
    \ft & ::= & \tty_1 \rightarrow \tty_2 & \mbox{(Function type)}\\
    \\
    p & ::= & (x_{1} : \tau_{1})~\ldots~(x_{n} : \tau_{n})  & \mbox{(\lstinline{let} or $\lambda$ pattern)} \\
    \hat{p} & ::= & (x_{1} : \utau_{1})~\ldots~(x_{n} : \utau_{n})  & \mbox{(Function pattern)} \\
    \\
    fun & ::= & \mbox{\Fun{$f$}{$\hat{p}$}{$\utau$}{$e$}} ~~& ~~\mbox{(Named function)} \\
    P & ::= & \mbox{$\epsilon$ ~ $|$ ~ $fun$ ~ $P$} & \mbox{(Program)} \\
    \\
    e & ::= & \mbox{\texttt{($x_{1},\ldots,x_{n}$)}} & \mbox{($n$-tuple)}\\
      & | & \mbox{\lstinline[mathescape]!let\ $p$ =\ $e$ in\ $b$!} & \mbox{(Let binding)}\\
      & | & k & \mbox{(Constant)} \\
      & | & x & \mbox{(Variable)} \\
      & | & x \odot{} x & \mbox{(Scalar binary operator)}\\
      & | & \mbox{\lstinline[mathescape]!if\ $x$ then\ $e$ else\ $e$!} & \mbox{(Branch)} \\
      & | & \mbox{\texttt{$x$[$x_{1},\ldots,x_{n}$]}} & \mbox{(Array indexing)}\\
      & | & \mbox{\lstinline[mathescape]!$x$ with [$x_{1},\ldots,x_{n}$] <-\ $x$!} &\mbox{(In-place update)}\\
      & | & \mbox{\lstinline[mathescape]!loop ($\hat{p}$)=($x_{1},\ldots,x_{n}$)!}  &  \mbox{(Loop)} \\
      &   & \mbox{\lstinline[mathescape]!\ for\ $x$ <\ $x$ do\ $e$!}\\
      & | & f~x_{1}~\ldots~x_{n} & \mbox{(Function call)}\\
      & | & \emph{op}~a_{1}~\ldots~a_{n} & \mbox{(Operator call)} \\
    a & ::= & x & \mbox{(Simple argument)} \\
      & |   & (x_{1}, \ldots, x_{n}) & \mbox{(Tuple argument)} \\
      & |   & (\Fn{\tty}{p}{e}) & \mbox{(Function argument)}
  \end{array}
  $$

  \caption{Grammar for the core Futhark IR.  Some compiler stages may
    impose additional constraints on the structure of ASTs, in
    particular by requiring size annotations to be present, or banning
    certain operations.  The definition of $op$ in particular may
    differ between stages.  Some constructs, such as \lstinline{while}
    loops, have been elided for simplicity.  The elided constructs
    would have no influence on the development of the thesis.}
  \label{fig:futhark-ir}
\end{figure*}

\Cref{fig:futhark-ir} shows the (simplified) IR of Futhark,
which will be used for all compiler-chapters of this thesis.  The
syntax is intentionally similar to the source language.  Notable
details include:
\begin{itemize}
\item Variable names are ranged over by $d$, $x$, $y$, and $z$, and we use
  $f$ to range over function names.
\item A variable may have a scalar type (i.e., \texttt{bool},
  \texttt{i32}, \texttt{f64}), or a multidimensional (regular) array
  type, such as $[][]\texttt{f64}$, which is the type of a matrix in
  which all rows have the same size.
\item Named and unnamed functions use a syntax similar to the source
  language, but unnamed functions may appear only as immediate
  arguments to SOACS, such as \lstinline{map}, \lstinline{reduce},
  \lstinline{filter}, and so on. Moreover, named functions may only be
  defined at top-level.  Recursion is not permitted.
\end{itemize}
Expressions ($e$) in the intermediate language are in A-normal
form~\cite{Sabry:1992:RPC:141478.141563}.  That is, all subexpressions
are variable names (except for the special cases \kw{loop}, \kw{if},
and \kw{let}).  Tuple-typed variables do not exist in the IR, except
as a syntactical construct for some operators.  A let-bound pattern
($p$) consists of one or more variables and associated types that bind
the result of an expression ($e$); intuitively equivalent to a tuple
pattern in the source language.  Types are syntactically
differentiated into two sorts: $\tau$, without uniqueness attributes,
and $\utau$, which permit an optional uniqueness attribute (an
asterisk).  Uniqueness attributes are only used when specifying the
parameter and return types of top-level functions and
\lstinline{for}-loops---patterns in \lstinline{let} bindings and
anonymous functions possess no uniqueness attributes.


\begin{figure}
\begin{tabular}{lp{58mm}}
  $\emph{op}~a_{1}~\ldots~a_{n}$ & Description \\ \hline
  \lstinline[mathescape]!size $c$ $x$! & Returns the size of dim $x$ of $x$, where $x$ must be a non-negative integer literal. \\
  \lstinline[mathescape]!iota $x$! & Returns the vector $[0,\ldots, x-1]$. \\
  \lstinline[mathescape]!replicate $y$ $x$! & Returns an array of rank one higher than $x$'s rank, containing an $y$-times replication of $x$.\\
  \lstinline[mathescape]!reshape ($y_{1},\ldots,y_{n}$) $x$! & Return an array of shape $y_{1}\times\ldots\times y_{n}$ containing the row-major elements of $x$.  It is a dynamic error if $x$ does not contain exactly $y_{1}\times\ldots\times y_{n}$ elements. \\
  \lstinline[mathescape]!rearrange ($c_{1}, \ldots, c_{n}$) $x$! & Rearrange the order of dimensions of the $n$-dimensional array $x$.  The reordering $c_{1}, \ldots, c_{n}$ must be a permutation of $1...n$. \\
  \lstinline[mathescape]!map $\lambda$ $x_{1}$ $\ldots$ $x_{n}$! & Apply the $n$-ary function $\lambda$ simultaneously to consecutive elements of $x_{1}\ldots{}x_{n}$, producing an array of the results.  If $\lambda$ returns $k$ values, $k$ arrays are returned. \\
  \lstinline[mathescape]!scatter $x$ $y$ $z$! & For all $i$, write $z[i]$ to position $y[j]$ in $x$.  Consumes $x$ and semantically returns a new array.  The result is unspecified if different values are written to the same position.  Out-of-bound writes have no effect. \\
  \lstinline[mathescape]!reduce $\lambda$ ($y_{1}, \ldots, y_{n}$) $x_{1}$ $\ldots$ $x_{n}$! &
                                                                                               Perform a reduction of arrays $x_{1}, \ldots, x_{n}$ via the $2n$-ary function $\lambda$, producing $n$ values.  The values $y_{1}, \ldots, y_{n}$ constitute a neutral argument for $\lambda$.
  \\
  \lstinline[mathescape]!scan $\lambda$ ($y_{1}, \ldots, y_{n}$) $x_{1}$ $\ldots$ $x_{n}$! &
                                                                                             Perform an inclusive prefix scan of arrays $x_{1}, \ldots, x_{n}$ via the $2n$-ary function $\lambda$, producing $n$ values.  The values $y_{1}, \ldots, y_{n}$ constitute a neutral argument for $\lambda$.
\end{tabular}
\caption{Description of SOAC operators.}
\label{fig:soacDesc}
\end{figure}


\begin{figure}[hbt]
\begin{tabular}{lcl}
\emph{op} & & \textrm{TySch}(\emph{op}) \\ \hline
  {\lstinline!size!} & : & $\forall\alpha.\texttt{i32} \rightarrow \alpha \rightarrow \texttt{i32}$ \\
  {\lstinline!iota!} & : & $\texttt{i32} \rightarrow []\texttt{i32}$ \\
  {\lstinline!replicate!} & : & $\forall\alpha.\texttt{i32} \rightarrow \alpha \rightarrow []\alpha$ \\
  \lstinline[mathescape]!reshape! & : & $\forall\alpha.(\texttt{i32}_{1}, \ldots, \texttt{i32}_{n})\rightarrow[]_{1}\cdots[]_{m}\alpha\rightarrow[]_{1}\cdots[]_{n}\alpha$ \\
  \lstinline[mathescape]!rearrange ($c_{1}, \ldots, c_{n}$)! & : & $\forall\alpha.[]_{1}\cdots[]_{n}\alpha\rightarrow[]_{1}\cdots[]_{n}\alpha$ \\
  {\lstinline!map!} & : & $\forall\bar{\alpha}^{(n)}\bar{\beta}^{(m)}. (\nseq{\alpha}{n} \rightarrow (\bar{\beta}^{(m)}))$ \\
          & & ~~~~~~~~~~~~~~~~~ $\rightarrow \nseq{[]\alpha_i}{n}$ \\
          & & ~~~~~~~~~~~~~~~~~ $\rightarrow ([]\beta_1,\ldots,[]\beta_m)$ \\
  {\lstinline!scatter!} & : & $\forall \nseq{\alpha}{n}.(\nseq{[]\alpha_{i}}{n})$\\
          & & ~~~~~~~~~~ $\rightarrow (\nseq{[]\texttt{i32}}{n})$\\
          & & ~~~~~~~~~~ $\rightarrow (\nseq{[]\alpha_i}{n})$\\
          & & ~~~~~~~~~~ $\rightarrow (\nseq{[]\beta_i}{n})$\\
  {\lstinline!reduce!} & : & $\forall\bar{\alpha}^{(n)}.(\nseq{\alpha}{n} \rightarrow \nseq{\alpha}{n} \rightarrow (\bar{\alpha}^{(n)}))$\\
          & & ~~~~~~~~~~ $\rightarrow (\nseq{\alpha}{n})$\\
          & & ~~~~~~~~~~ $\rightarrow \nseq{[]\alpha_i}{n}$\\
          & & ~~~~~~~~~~ $\rightarrow (\nseq{\alpha}{m})$ \\
  {\lstinline!scan!} & : & $\forall\bar{\alpha}^{(n)}.(\nseq{\alpha}{n} \rightarrow \nseq{\alpha}{n} \rightarrow (\bar{\alpha}^{(n)}))$\\
          & & ~~~~~~~~~~ $\rightarrow (\nseq{\alpha}{n})$\\
          & & ~~~~~~~~~~ $\rightarrow \nseq{[]\alpha_i}{n}$\\
          & & ~~~~~~~~~~ $\rightarrow (\nseq{[]\alpha_i}{m})$ \\
  {\lstinline!filter!} & : & $\forall\bar{\alpha}^{(n)}.(\alpha_1 \rightarrow \ldots \rightarrow \alpha_n \rightarrow \texttt{bool})$ \\
          & & ~~~~~~~~~~ $\rightarrow \nseq{[]\alpha_i}{n}$ \\
          & & ~~~~~~~~~~ $\rightarrow (\nseq{[]\alpha_i}{n})$ \\
\end{tabular}
\caption{Size-agnostic type schemes for operators, including various SOACs.}
\label{fig:soacType}
\end{figure}

An \textit{array operator} ($op$) is one of several constructs that
operate on arrays.  These include SOACs, but also such operations as
\lstinline{iota} and \lstinline{reshape}.  An array operator can be
applied to multiple arguments ($a$), where an argument can be a
variable in scope, but may also be a tuple of variables, or an
anonymous function.

For giving concise types to operators, we use a notion of \emph{extended
types} that supports polymorphism in types and for which arguments to
functions may themselves be functions:

\begin{tabular}{lrl}
$\etau$ & $::=$ & $\alpha~|~(\tau_{1}, \ldots, \tau_{n})~|~\etau_1 \rightarrow \etau_2~|~\etau \rightarrow \etau$ \\
$\opty$ & $::=$ & $\forall\bar{\alpha}.\etau$
\end{tabular}

\noindent Extended types ($\ubar{\tau}$) and extended type schemes
($\ubar{\sigma}$) are used only for the treatment of operators and we
shall be implicit about converting types and type schemes to and from
their extended counter parts.  We treat the single-element tuple
$(\tau)$ as equivalent to $\tau$.  A \emph{substitution} ($S$) is a
mapping from type variables to extended types. Applying a substitution
$S$ to some object $B$, written $S(B)$, has the effect of
simultaneously applying $S$ to type variables in $B$ (being the
identity outside its domain). An extended type $\etau'$ is \emph{an
  instance of} an extended type scheme
$\opty = \forall \vec{\alpha}.\etau$, written $\opty \geq \etau'$,
if there exists a substitution $S$ such that $S(\etau) = \etau'$.

Type schemes for operators, including a representative subset of the
SOAC operators, are given in \Cref{fig:soacType}, and an
informal description of them is given in \Cref{fig:soacDesc}.
The SOACs of the intermediate language (e.g., \lstinline{map}) are the
tuple-of-array version of the user language SOACs. The SOACs of the
intermediate language receive an arbitrary number of array arguments
and produce a tuple of arrays. The semantics of a SOAC operator can be
intuitively understood as a composition between \lstinline{unzip}, the
user-language SOAC (e.g., \lstinline{map}), and \lstinline{zip}, where the
unnamed function is suitably modified to work with the flat sequence
of array arguments.

In the type rules, we permit the implicit transformation of uniqueness
types and patterns $\utau/\utty/\hat{p}$ to their corresponding
$\tau/\tty/p$, where uniqueness attributes have simply been removed.
This simplifies the type rules, which can then be stated separate from
the rules for checking uniqueness properties.

Type environments ($\Gamma$) are finite
maps from program variables to types or function types. When $\Gamma$
is a type environment and $p$ is a pattern
$\tau_1~x_1,\cdots,\tau_n~x_n$, we write $\Gamma,p$ to denote the
typing environment $\Gamma,x_1:\tau_1,\cdots,x_n:\tau_n$.

Typing rules for the source language are given in
Figures~\ref{fig:srcTypeRules} and \ref{fig:srcTypeRulesExps}.  The
rules allow inferences among sentences of the following forms:

\begin{description}
\item[$\vdp p : \tty$,] read ``the pattern $p$ is matched by
  expressions of tuple type $\tty$.''
\item[$\vdhp \hat{p} : \utty$,] read ``the pattern with uniqueness
  $\hat{p}$ is matched by expressions of tuple type $\tty$.''
\item[$\Gamma \vda a : \tau / \ft$,] read ``under the assumptions
$\Gamma$, the operator argument $a$ has type $\tau$ or function type
$\ft$.''
\item[$\Gamma \vd b : \rho$,] read ``under the assumptions $\Gamma$,
  the expression $e$ has tuple type $\rho$.''
\item[$\Gamma \vdP P : \rho$,] read ``under the assumptions
$\Gamma$, the program $P$ has tuple type $\rho$.''
\end{description}

\begin{figure}[bt]
\sempart{Patterns}{\vdp p : \tty}

\fracc{}{\vdp (x_1: \tau_1, \cdots, x_n: \tau_n) : (\tau_1, \cdots, \tau_n)}

\sempart{Operator arguments}{\vda a : \tty / \ft}

\fracc{}{\Gamma \vda x : \Gamma(x)}

\fracc{\vdp p : \tty \sp \Gamma,p \vd b : \tty'}{\Gamma \vda \Fn{\tty'}{p}{b}: \tty \rightarrow \tty'}

\fracc{}{\vdp (x_1, \ldots, x_n) : (\Gamma(x_1), \ldots, \Gamma(x_n))}

\sempart{Parameters}{\vdhp \hat{p} : \utty}

\fracc{}{\vdhp (x_1: \utau_1, \cdots, x_n: \utau_n) : (\utau_1, \cdots, \utau_n)}

\sempart{Programs}{\Gamma \vdP P}

\fracc{\vdhp \hat{p} : (\tau_1, \ldots, \tau_n) \sp \Gamma,\hat{p} \vd e : \tty \\
\Gamma(g) = \tau_1 \rightarrow \ldots \rightarrow \tau_n \rightarrow \tau_{n+1}  \sp  \Gamma \vdP P}
{\Gamma \vdP \Fun{$f$}{$\hat{p}$}{$\utty_2$}{$e$}~P}

\fracc{}{\Gamma \vdP \epsilon}

\caption{Typing rules for Futhark's core IR, excluding expressions.}
\label{fig:srcTypeRules}
\end{figure}

\begin{figure}[bt]
\sempart{Expressions}{\Gamma \vd e : \rho}
\fracc{}{\Gamma \vd (x_1,\cdots,x_n) : (\Gamma(x_1),\cdots,\Gamma(x_n))}

\fracc{\Gamma(y) = []\tau \sp \Gamma(s) = \texttt{i32} \sp \Gamma,x:\tau \vd e : \rho}
{\Gamma \vd \Let{x:\tau}{y[s]}{e} : \rho}

\fracc{\textrm{ConstType}(\emph{ct}) = \tau \sp \Gamma, x:\tau \vd e : \rho}{\Gamma \vd \Let{x:\tau}{\emph{c}}{e} : \rho}

\fracc{\Gamma \vda a_i : \etau_i \sp i \in \Set{1, \ldots, n}\\
\textrm{TySch}(\emph{op}) \geq (\bar{\etau}^{(n)}) \rightarrow \rho \sp \vdp p : \rho \sp \Gamma,p \vd e : \rho'}
{\Gamma \vd \Let{p}{\emph{op}~\bar{a}^{(n)}}{e} : \rho'}

\fracc{\Gamma(x_i) = \tau_i \sp i \in \Set{1, \ldots, n} \\
\textrm{lookup}_{\textrm{fun}}(f) = \tau_1 \rightarrow \cdots \rightarrow \tau_n \rightarrow \rho \sp \vdp p : \rho \sp \Gamma,p \vd e : \rho'}
{\Gamma \vd \Let{p}{f~x_{1}~\ldots~x_n}{e} : \rho'}

\fracc{\Gamma(s) = \texttt{bool} \sp \Gamma \vd e_1 : \rho \sp \Gamma \vd e_2 : \rho \\
\vdp p : \rho \sp \Gamma,p \vd e_3 : \rho'}
{\Gamma \vd \Let{p}{\If{s}{e_1}{e_2}}{e_3} : \rho'}

\caption{Typing rules for expressions in Futhark's core IR.  For
  simplicity, we assume that the only expressions that occur after
  ``\kw{in}'' is another \kw{let}-binding, a variable name, or a
  tuple, meaning that all other expressions are always bound to a
  pattern.}
\label{fig:srcTypeRulesExps}
\end{figure}

\fixme{Some rules are missing.}

\section{Checking Uniqueness Types}
\label{sec:uniqueness-formalism}

\subsection{Alias Analysis}

\newcommand{\expAliases}[3]{#1 \vdash #2 \Rightarrow #3}
\newcommand{\aliases}[1]{\textrm{aliases}(#1)}
\newcommand{\seqOccurences}[3]{#1 \gg #2 : #3}

We perform alias analysis on a program that we assume to be otherwise
type-correct.  Our presentation uses an inference rule-based approach
%similar to the one usually used for type systems.
in which the central judgment takes the form %as follows:
$\expAliases{\Sigma}{e}{\langle \sigma_{1}, \ldots, \sigma_{n} \rangle}$, which
asserts that, within the context $\Sigma$, the
expression $e$ produces $n$ values, where value number $i$ has the
\textit{alias set} $\sigma_{i}$.  An alias set is a subset of the
variable names in scope, and indicates which variables an array value
(or variable) may share elements with.
%(Alias sets for non-array variables are meaningless in this context).
The context $\Sigma$ maps variables in scope
to their aliasing sets.

\begin{figure}
\sempart{Expression aliases}{\expAliases{\Sigma}{e}{\langle \sigma_{1}, \ldots, \sigma_{n} \rangle}}

\begin{equation*}
\inference{
}{
  \expAliases{\Sigma}{(x_{1}, \ldots, x_{n})}{\langle \{ x_{1} \} \cup \Sigma(x_{1}), \ldots,  \{ x_{n} \} \cup \Sigma(x_{n}) \rangle}
}
\tagsc{Alias-Tuple}
\end{equation*}

\begin{equation*}
\inference{
  \expAliases{\Sigma}{e_{1}}{\langle \nseq{\sigma}{n} \rangle}
  \\
  \expAliases{\Sigma, x_{i} \mapsto \sigma_{i}}{e_{2}}{\langle \nseq{\sigma'}{n} \rangle}
}{
  \expAliases{\Sigma}{\Let{\nseq{(x:\tau)}{n}}{e_{1}}{e_{2}}}
  {\langle \nseq{\sigma'}{n} \rangle \setminus \{\nseq{x}{n}\}}
}
\tagsc{Alias-LetPat}
\end{equation*}

\begin{equation*}
\inference{
}{
  \expAliases{\Sigma}{c}{\langle \emptyset \rangle}
}
\tagsc{Alias-Const}
\end{equation*}

\begin{equation*}
\inference{
}{
  \expAliases{\Sigma}{x}{\langle \{ x \} \cup \Sigma(x) \rangle}
}
\tagsc{Alias-Var}
\end{equation*}

\begin{equation*}
\inference{
}{
  \expAliases{\Sigma}{x \oplus y}{\langle \emptyset \rangle}
}
\tagsc{Alias-ScalarBinOp}
\end{equation*}

\begin{equation*}
\inference{
  \expAliases{\Sigma}{e_{2}}{\langle s^{2}_{1}, \ldots, s^{2}_{n}\rangle}
  &
  \expAliases{\Sigma}{e_{3}}{\langle s^{3}_{1}, \ldots, s^{3}_{n}\rangle}
}{
  \expAliases{\Sigma}{\text{\lstinline[mathescape]!if\ $x_{1}$ then\ $e_{2}$ else\ $e_{3}$!}}
  {\langle s^{2}_{1} \cup s^{3}_{1}, \ldots, s^{2}_{n} \cup s^{3}_{n}\rangle}
}
\tagsc{Alias-If}
\end{equation*}

\begin{equation*}
\inference{
  \textrm{$x$ is of rank $n$}
}{
  \expAliases{\Sigma}{\textsf{$x$[$\nseq{x}{n}$]}}{\langle \emptyset \rangle}
}
\tagsc{Alias-IndexArray}
\end{equation*}

\begin{equation*}
\inference{
  \textrm{$x$ is of rank $>n$}
}{
  \expAliases{\Sigma}{\textsf{$x$[$\nseq{x}{n}$]}}{\langle \{ x \} \cup \Sigma(x) \rangle}
}
\tagsc{Alias-SliceArray}
\end{equation*}

\begin{equation*}
\inference{
}{
\expAliases{\Sigma}{
  \text{\lstinline[mathescape]!$x_{a}$ with [$\nseq{x}{n}$] <-\ $x_{x}$!}
}{\langle \Sigma(x_{a}) \rangle}
}
\tagsc{Alias-Update}
\end{equation*}

\caption{Aliasing rules for simple expressions.}
\label{fig:aliasing-rules-1}
\end{figure}

\fixme{only 1 parameter in loop case}
\fixme{only 1 return in apply case}
\begin{figure}
\sempart{Expression aliases (continued)}{\expAliases{\Sigma}{e}{\langle \sigma_{1}, \ldots, \sigma_{n} \rangle}}

\begin{equation*}
  \inference{
    \textrm{$\tau$ is not of form \texttt{*}$\tau'$}
    \\
    \expAliases{\Sigma}{x_{1}}{\langle \sigma \rangle}
    \sp
    \expAliases{\Sigma, x_{1} \mapsto \sigma}{e}{\langle \sigma' \rangle}
}{
\expAliases{\Sigma}
{\text{\lstinline[mathescape]!loop ($x_{1}: \tau$) =\ $y_{2}$ for\ $z_{1}$ <\ $z_{2}$ do\ $e$!}}
{\langle \sigma' \setminus \{x_{1}\} \rangle}
}\tagsc{Alias-DoLoop-Nonunique}
\end{equation*}

\begin{equation*}
  \inference{
}{
\expAliases{\Sigma}
{\text{\lstinline[mathescape]!loop ($x_{1}: \texttt{*}\tau$) =\ $y_{2}$ for\ $z_{1}$ <\ $z_{2}$ do\ $e$!}}
{\langle \emptyset  \rangle}
}\tagsc{Alias-DoLoop-Unique}
\end{equation*}

\begin{equation*}
\inference{
  \textrm{lookup}_{\textrm{fun}}(f) = \utau_{1}\rightarrow\cdots\rightarrow\utau_{n}\rightarrow\tau
  \\
  \expAliases{\Sigma}{x_{i}}{\langle \sigma_{i} \rangle}
  \quad
  \sigma = \bigcup_{\textrm{$\tau_{i}$ is not of form \texttt{*}$\tau'$}} \sigma_{i}
}{
  \expAliases{\Sigma}{\textsf{$f$ $x_{1}$ \ldots{} $x_{n}$}}
  {\langle \sigma \rangle}
}\tagsc{Alias-Apply-Nonunique}
\end{equation*}

\begin{equation*}
\inference{
  \textrm{lookup}_{\textrm{fun}}(f) = \utau_{1}\rightarrow\cdots\rightarrow\utau_{n}\rightarrow\texttt{*}\tau
}{
  \expAliases{\Sigma}{\textsf{$f$ $x_{1}$ \ldots{} $x_{n}$}}
  {\langle \emptyset \rangle}
}\tagsc{Alias-Apply-Unique}
\end{equation*}

\begin{equation*}
\inference{
}{
  \expAliases{\Sigma}{\text{\lstinline[mathescape]!iota\ $x$!}}{\langle \emptyset \rangle}
}\tagsc{Alias-Iota}
\end{equation*}

\begin{equation*}
\inference{
}{
  \expAliases{\Sigma}{\text{\lstinline[mathescape]!replicate\ $x$\ $y$!}}{\langle \emptyset \rangle}
}\tagsc{Alias-Replicate}
\end{equation*}

\begin{equation*}
  \inference{
    \expAliases{\Sigma}{y}{\sigma}
}{
  \expAliases{\Sigma}{\text{\lstinline[mathescape]!reshape\ $(x_{1},\ldots,x_{n})$\ $y$!}}{\langle \sigma \rangle}
}\tagsc{Alias-Reshape}
\end{equation*}

\begin{equation*}
\inference{
    \expAliases{\Sigma}{y}{\sigma}
}{
  \expAliases{\Sigma}{\text{\lstinline[mathescape]!rearrange\ ($c_{i},\ldots,c_{n}$)\ $y$!}}{\langle \sigma \rangle}
}\tagsc{Alias-Rearrange}
\end{equation*}

\begin{equation*}
\inference{
}{
  \expAliases{\Sigma}{\text{\lstinline[mathescape]!map ($\lambda p$:\ $(\nseq{\tau}{m}) \rightarrow e$)\ $\nseq{x}{n}$!}}
  {\langle \nseq{\emptyset}{m} \rangle}
}
\tagsc{Alias-Map}
\end{equation*}

\begin{equation*}
\inference{
}{
  \expAliases{\Sigma}{\text{\lstinline[mathescape]!reduce\ $a$\ ($\nseq{x}{n}$)\ $\nseq{x}{n}$!}}
  {\langle \nseq{\emptyset}{n} \rangle}
}
\tagsc{Alias-Reduce}
\end{equation*}

\begin{equation*}
\inference{
}{
  \expAliases{\Sigma}{\text{\lstinline[mathescape]!scan\ $a$\ ($\nseq{x}{n}$)\ $\nseq{x}{n}$!}}
  {\langle \nseq{\emptyset}{n} \rangle}
}
\tagsc{Alias-Scan}
\end{equation*}

\begin{equation*}
\inference{
}{
  \expAliases{\Sigma}{\text{\lstinline[mathescape]!filter\ $a$\ $\nseq{x}{n}$!}}
  {\langle \nseq{\emptyset}{n} \rangle}
}
\tagsc{Alias-Filter}
\end{equation*}

\caption{More aliasing rules.}
\label{fig:aliasing-rules-2}
\end{figure}

\fixme{not space reasons}
The aliasing rules are listed in Figures~\ref{fig:aliasing-rules-1}
and \ref{fig:aliasing-rules-2}, although for space reasons, some are
left out.  The \textsc{Alias-Var}-rule defines the aliases of a
variable expression to be the alias set of the variable joined by the
name of the variable itself - this is because $v \notin \Sigma(v)$, as
can be seen by \textsc{Alias-LetPat}.  Alias sets for values produced
by SOACs such as \lstinline{map} are empty.  We can imagine the arrays
produced as \textit{fresh}, although the compiler is of course free to
reuse existing memory if it can do so safely.  The
\textsc{Alias-IndexArray} rule tells us that a scalar read from an
array does not alias its origin array, but \textsc{Alias-SliceArray}
dictates that an array slice does, which fits the implementation
intuition.
% about how such operations might be implemented.


The most interesting aliasing rules are the ones for function calls
(\textsc{Alias-Apply-Nonunique} and \textsc{Alias-Apply-Unique}).
Since our alias analysis is intra-procedural, we are forced to be
conservative.  There are two rules, corresponding to functions
returning unique and non-unique arrays, respectively. When the result
is unique the alias set is empty, otherwise the result conservatively
aliases all non-unique parameters.

\subsection{In-Place Update Checking}

\newcommand{\inPlaceSafe}[3]{#1\ \rhd \langle #2, #3 \rangle}
\newcommand{\consumedOK}[3]{#1 \vdash #2 \triangle #3}

In our implementation, alias computation and in-place update checking
is performed at the same time, but is split here for expository
purposes.  Let $\aliases{v}$ the alias set of the variable $v$.  We
denote by $\mathcal{O}$ the set of the variables \textit{observed}
(used) in expression $e$, and by $\mathcal{C}$ the set of variables
\textit{consumed} through function calls and in-place updates.
Together, the pair $\langle\mathcal{C},\mathcal{O}\rangle$ is called
an \textit{occurrence trace}.

\Cref{fig:uniqueness-rules} defines a \textit{sequencing} judgment
between two occurrence traces, which takes the form $\seqOccurences
  {\langle\mathcal{C}_{1},\mathcal{O}_{1}\rangle}
  {\langle\mathcal{C}_{2},\mathcal{O}_{2}\rangle}
  {\langle\mathcal{C}_{3},\mathcal{O}_{3}\rangle}$
%We also define a judgment for \textit{sequencing} two occurrence
%traces on figure \ref{fig:aliasing-rules}.  The sequencing judgment
%\begin{equation*}\boxed{
%  \seqOccurences
%  {\langle\mathcal{C}_{1},\mathcal{O}_{1}\rangle}
%  {\langle\mathcal{C}_{2},\mathcal{O}_{2}\rangle}
%  {\langle\mathcal{C}_{3},\mathcal{O}_{3}\rangle}
%}\end{equation*}
and which can be derived if and only if it is acceptable for
$\langle\mathcal{C}_{1},\mathcal{O}_{1}\rangle$ to happen first, then
$\langle\mathcal{C}_{2},\mathcal{O}_{2}\rangle$, giving the combined
occurrence trace $\langle\mathcal{C}_{3},\mathcal{O}_3\rangle$.  The
formulation as a judgment is because sequencing is sometimes not
derivable---for example in the case where an array is used after  %this situation corresponds to
it has been consumed.  The judgment is defined by a single inference
rule, which states that two occurrence traces can be sequentialized if
and only if no array consumed in the left-hand trace is used in the
right-hand trace.

Some of the inference rules for checking if an expression $e$ is
functionally safe with respect to in-place updates are shown in
\Cref{fig:uniqueness-rules}, where the central judgment is
$\inPlaceSafe{e}{\mathcal{C}}{\mathcal{O}}$.
%\begin{equation*}
%\boxed{
%\inPlaceSafe{e}{\mathcal{C}}{\mathcal{O}}
%}
%\end{equation*}
%

The rule for in-place update
\lstinline[mathescape]{$v_{a}$ with [$\nseq{v}{n}$] <- $v_{v}$} gives rise to an
occurrence trace indicating that we have \textit{observed} $v_{v}$ and
\textit{consumed} $v_{a}$.  Indices $\nseq{v}{n}$ are ignored as
they are necessarily scalar variables and cannot be consumed.

Another case is checking the safety of a \lstinline{map} expression.
We do not wish to permit the function of a \lstinline{map} to consume
any array bound outside of it, as that would imply the array is
consumed once for every iteration of the \lstinline{map}.  However,
the function may consume its parameters, which should be seen as the
\lstinline{map} expression as a whole consuming the corresponding
input array.  This restriction also preserves the parallel semantics
of \lstinline{map}, because different rows of a matrix can be safely
updated in parallel.  An example can be seen on
\Cref{fig:map-in-place-update}, which shows an in-place update
nested inside an array.  To express this restriction, we define an
auxiliary judgment
$\consumedOK {\mathcal{P}}
{\langle\mathcal{C}_{1},\mathcal{O}_{1}\rangle}
{\langle\mathcal{C}_{2},\mathcal{O}_{2}\rangle}$.
%
%:\begin{equation*} \boxed{ \consumedOK {\mathcal{P}}
%  {\langle\mathcal{C}_{1},\mathcal{O}_{1}\rangle}
%  {\langle\mathcal{C}_{2},\mathcal{O}_{2}\rangle} }
%\end{equation*}
Here, $\mathcal{P}$ is a mapping from parameter names to alias sets.
Any variable $v$ in $\mathcal{O}_{1}$ that has a mapping in
$\mathcal{P}$ is replaced with $\mathcal{P}[v]$ to produce
$\mathcal{O}_{2}$.  If no such mapping exists, $v$ is simply included
in $\mathcal{O}_{2}$.  Similarly, any variable $v$ in
$\mathcal{C}_{1}$ that has a mapping in $\mathcal{P}$ is replaced with
the variables in the set $\mathcal{P}[v]$ (taking the union of all
such replacements), producing $\mathcal{C}_{2}$.  However, if $v$ does
not have such a mapping, the judgment is not derivable.
%This
%corresponds to the body of the function consuming something that is not
%a parameter.
%
The precise inference rules are shown on
\Cref{fig:parameter-consumption}.
%
Do-loops and function declarations can be checked for safety in a
similar way.  A function is safe with respect to in-place updates if
its body consumes only those of the function's parameters that are
unique.

\begin{figure}
\sempart{Validity of sequencing}{\seqOccurences{\langle\mathcal{C}_{1},\mathcal{O}_{1}\rangle}{\langle\mathcal{C}_{2},\mathcal{O}_{2}\rangle}{\langle\mathcal{C}_{3},\mathcal{O}_{3}\rangle}}

\begin{equation*}
\inference{
  (\mathcal{O}_{2} \cup \mathcal{C}_{2}) \cap \mathcal{C}_{1} = \emptyset
}{
  \seqOccurences
  {\langle\mathcal{C}_{1},\mathcal{O}_{1}\rangle}
  {\langle\mathcal{C}_{2},\mathcal{O}_{2}\rangle}
  {\langle\mathcal{C}_{1}\cup\mathcal{C}_{2},\mathcal{O}_{1}\cup\mathcal{O}_{2}\rangle}
}\tagsc{Occurence-Seq}
\end{equation*}

\sempart{Uniqueness Safety Rules}{\inPlaceSafe{e}{\mathcal{C}}{\mathcal{O}}}

\begin{equation*}
\inference{
}{
\inPlaceSafe{v}{\emptyset}{\aliases{v}}
}
\tagsc{Safe-Var}
\end{equation*}

\begin{equation*}
\inference{
}{
\inPlaceSafe{k}{\emptyset}{\emptyset}
}
\tagsc{Safe-Const}
\end{equation*}

\begin{equation*}
\inference{
  \inPlaceSafe{e_{1}}{\mathcal{C}_{1}}{\mathcal{O}_{1}}
  &
  \inPlaceSafe{e_{2}}{\mathcal{C}_{2}}{\mathcal{O}_{2}}
  \\
  \seqOccurences
  {\langle\mathcal{C}_{1},\mathcal{O}_{1}\rangle}
  {\langle\mathcal{C}_{2},\mathcal{O}_{2}\rangle}
  {\langle\mathcal{C}_{3},\mathcal{O}_{3}\rangle}
}{
  \inPlaceSafe{\Let{v_{1}~\ldots~v_{n}}{e_{1}}{e_{2}}}
  {\mathcal{C}_{3}}{\mathcal{O}_{3}\rangle}
}
\tagsc{Safe-LetPat}
\end{equation*}

\begin{equation*}
\inference{
  \inPlaceSafe{v_{1}}{\mathcal{C}_{1}}{\mathcal{O}_{1}}
  &
  \inPlaceSafe{e_{2}}{\mathcal{C}_{2}}{\mathcal{O}_{2}}
  &
  \inPlaceSafe{e_{3}}{\mathcal{C}_{3}}{\mathcal{O}_{3}}
  \\
  \seqOccurences
  {\langle\mathcal{C}_{1},\mathcal{O}_{1}\rangle}
  {\langle\mathcal{C}_{2},\mathcal{O}_{2}\rangle}
  {\langle\mathcal{C}_{2}',\mathcal{O}_{2}'\rangle}
  \\
  \seqOccurences
  {\langle\mathcal{C}_{1},\mathcal{O}_{1}\rangle}
  {\langle\mathcal{C}_{3},\mathcal{O}_{3}\rangle}
  {\langle\mathcal{C}_{3}',\mathcal{O}_{3}'\rangle}
}{
  \inPlaceSafe{\text{\lstinline[mathescape]!if\ $v_{1}$ then\ $e_{2}$ else\ $e_{3}$!}}
  {\mathcal{C}'_{2}\cup\mathcal{C}'_{3}}
  {\mathcal{O}'_{2}\cup\mathcal{O}'_{3}}
}
\tagsc{Safe-If}
\end{equation*}

\begin{equation*}
\inference{
}{
  \inPlaceSafe{
    \text{\lstinline[mathescape]!$v_{a}$ with [$\nseq{v}{n}$] <- $v_{v}$!}
  }{\aliases{v_{a}}}{\aliases{v_{n}}}
}\tagsc{Safe-Update}
\end{equation*}

\begin{equation*}
\inference{
  \inPlaceSafe{e_{b}}{\mathcal{C}}{\mathcal{O}}
  \\
  \consumedOK
  {\nseq{p_{i} \mapsto \aliases{v_{i}}}{n}}
  {\langle\mathcal{C},\mathcal{O}\rangle}
  {\langle\mathcal{C}',\mathcal{O}'\rangle}
}{
  \inPlaceSafe
  {\text{\lstinline[mathescape]!map ($\fn\nseq{p}{n}$: $\nseq{t}{m} \rightarrow e_{b}$) $\nseq{v}{n}$!}}
  {\mathcal{C}'}{\mathcal{O}'}
}
\tagsc{Safe-Map}
\end{equation*}

  \caption{Checking safety of consumption.}
  \label{fig:uniqueness-rules}
\end{figure}

\begin{figure}
\begin{equation*}
\boxed{
  \consumedOK
  {\mathcal{P}}
  {\langle\mathcal{C}_{1},\mathcal{O}_{1}\rangle}
  {\langle\mathcal{C}_{2},\mathcal{O}_{2}\rangle}
}
\end{equation*}

\begin{equation*}
\inference{
}{
\mathcal{P} \vdash \langle \emptyset, \emptyset \rangle \triangle \langle \emptyset, \emptyset \rangle
} \tagsc{Observe-BaseCase}
\end{equation*}

\begin{equation*}
\inference{
v \in \mathcal{P} \qquad \mathcal{P} \vdash \langle \emptyset, \mathcal{O} \rangle \triangle \langle \emptyset, \mathcal{O}' \rangle
}
{
\mathcal{P} \vdash \langle \emptyset, \{ v \} \cup \mathcal{O} \rangle \triangle \langle \emptyset, \mathcal{P}[v] \cup \mathcal{O}' \rangle
}\tagsc{Observe-Param}
\end{equation*}

\begin{equation*}
\inference{
\neg(v \in \mathcal{P}) \qquad \mathcal{P} \vdash \langle \emptyset, \mathcal{O} \rangle \triangle \langle \emptyset, \mathcal{O}' \rangle
}
{
\mathcal{P} \vdash \langle \emptyset, \{ v \} \cup \mathcal{O} \rangle \triangle \langle \emptyset, \{ v \} \cup \mathcal{O}' \rangle
}\tagsc{Observe-NonParam}
\end{equation*}

\begin{equation*}
\inference{
v \in \mathcal{P} \qquad \mathcal{P} \vdash \langle \mathcal{C}, \mathcal{O} \rangle \triangle \langle \mathcal{C}', \mathcal{O}' \rangle
}
{
\mathcal{P} \vdash \langle \{ v \} \cup \mathcal{C}, \mathcal{O} \rangle \triangle \langle \mathcal{P}[v] \cup \mathcal{C}', \mathcal{O}' \rangle
}\tagsc{Observe-NonParam}
\end{equation*}
  \caption{Checking parameter consumption.}
  \label{fig:parameter-consumption}
\end{figure}


\begin{figure}
\begin{lstlisting}
  -- This one is OK and considered to consume 'as'.
  let bs = map (\(a) -> !a with [0] <- 2!) as
  let d  = iota m
  -- This one is NOT safe, since d is not a formal parameter.
  let cs = map (\(i) -> @d with [i] <- 2@) (iota n)
\end{lstlisting}
  \caption{Examples of \lstinline{map}s with in-place updates.}
  \label{fig:map-in-place-update}
\end{figure}

%%% Local Variables:
%%% mode: latex
%%% TeX-master: "thesis"
%%% End:
