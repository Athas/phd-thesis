\newcommand{\zip}{\mathrm{zip}}
\newcommand{\unzip}{\mathrm{unzip}}
\newcommand{\map}{\mathrm{map}}
\newcommand{\reduce}{\mathrm{reduce}}
\newcommand{\scan}{\mathrm{scan}}
\newcommand{\ascan}{\mathrm{ascan}}
\newcommand{\sscan}{\mathrm{sscan}}
\newcommand{\inj}{\mathrm{inj}}
\newcommand{\rep}{\mathrm{rep}}
\newcommand{\streamFold}{\mathrm{sFold}}
\newcommand{\fold}{\mathrm{fold}}
\newcommand{\seg}[1]{\mathbf{#1}}
\newcommand{\tto}{\,\Rightarrow\,}
\newcommand{\iso}{\,\Leftrightarrow\,}
\newcommand{\nat}{\mathbf{N}}
\newcommand{\concat}{\#}
\newcommand{\idd}{\mathrm{id}}
\newcommand{\Id}{\mathrm{id}}
\newcommand{\foldl}{\mathrm{foldl}}

\chapter{An Array Calculus}

The theoretical foundation of Futhark are the list homomorphisms of
Bird and Meertens~\cite{Bird2}, realised in the form of purely
functional parallel second-order array combinators (SOACs). Their rich
equational theory for semantics-preserving transformations are
employed in Futhark for fusion, streaming, and flattening of
parallelism.  This chapter discusses a simple array combinator
calculus that I will use to convey the intuition behind certain useful
transformations on arrays and, more importantly, functions that
operate on arrays.  The calculus presented here is simpler than the
full Futhark programming language, yet also more flexible.

\section{Array Combinator Calculus}
\label{sec:arraycombinators}

\begin{figure}
  \centering
$$
\begin{array}{lrr}
  \alpha,\beta,\tau & ::= & \mbox{\texttt{t}}\ |\ (\tau_1, \ldots, \tau_n)\ |\ [n]\tau \\

  v & ::= & \mbox{\texttt{v}}\ |\ (v_{1}, \ldots, v_{n})\ |\ [v_{1}, \ldots, v_{n}] \\

  \hat{\alpha},\hat{\beta},\hat{\tau} &::=& \tau\ |\ \hat{\alpha}\rightarrow\hat{\beta}\ |\ \Pi n.\hat{\tau} \\

  e &::=& \mbox{\texttt{v}}\ |\ (e_{1}, \ldots, e_{n})\ |\ [e_{1}, \ldots, e_{n}]\ |\ e_1~e_2\ |\ \lambda x.e
\end{array}
$$
  \caption{Syntax of the Array Combinator Calculus.}
  \label{fig:arraycalculus_syntax}
\end{figure}

The array combinator calculus, a modified lambda calculus, describes
terms that operate on array values.  The syntax is summarised on
Figure~\ref{fig:arraycalculus_syntax}.  We make a distinction between
\textit{ground values}
$v$, which cannot contain functionals, and \textit{terms}
$e$, which can.  An array value is written as a comma-separated
sequence of values enclosed in square brackets, as in $[v_{1}, v_{2},
v_{3}]$.  Arrays may contain scalar values written as \texttt{t}
(which are not important for the calculus), other arrays, or tuples of
values.  A tuple is written as a comma-separated sequence of values
enclosed in parentheses.

Array values are classified by array types.  We write $[n]\tau$ to
denote the type of arrays with $n$ elements of type $\tau$.  The tuple
of a tuple is written as the types of its components separated by
commas and enclosed in parentheses.  Primitive (non-array) types are
written as $t$.  In contrast to previously published parallel array
formalisms such as MOA~\cite{moa}, singleton dimensions of arrays are
significant: $[1][n]\tau$ and $[n][1]\tau$ are distinct types.
Likewise, a singleton array $[1]\tau$ is distinct from a scalar of
type $\tau$.  Higher-order types for classifying functions are
described later.

\newcommand{\convV}[1]{\mathcal{C}_{\mathcal{V}}(#1)}
\newcommand{\convT}[1]{\mathcal{C}_{\mathcal{T}}(#1)}

We treat the zip/unzip-isomorphic types
$[n](\tau_1, \ldots, \tau_k) \cong [n]\tau_1 , \ldots , [n]\tau_k$ as
interchangeable in any context.  This is justified as any array of
tuples can be converted into a corresponding tuple of arrays, as shown
on Figure~\ref{fig:calc-tuple-transform}.  The function $\convV{v}$
rewrites a value $v$ such that no arrays of tuples appear.  Likewise,
the function $\convT{\tau}$ rewrites a type.  Intuitively, the
rewriting considers an array of tuples as an augmented array with an
extra dimension, followed by transposing out that extra dimension.
Actually constructing this array would not be well-typed, as all
elements of an array must have the same type, whereas tuples can
contain elements of differing types.

\begin{figure}
  \centering
  \begin{align*}
    &\convV{[(v_{(1,1)}, \ldots, v_{(1,m)}),\ldots,(v_{(n,1)}, \ldots, v_{(n,m)})]} =\\
    &\quad (\convV{[v_{(1,1)},\ldots,v_{(n,1)}]},\ldots,\convV{[v_{(1,m)},\ldots,v_{(n,m)}]})\\
    \\
    &\convT{[n](\tau_{1},\ldots,\tau_{n})} =\\
    &\quad(\convT{[n]\tau_{1}},\ldots,\convT{[n]\tau_{n}}) \\
  \end{align*}
  \caption{Converting arrays of tuples to tuples of arrays.}
  \label{fig:calc-tuple-transform}
\end{figure}

We first describe the basic SOACs and later introduce
\textit{streaming combinators}. The basic SOACs include (i)
\lstinline{map}, which constructs an array by applying its function argument to
each element of the input array, (ii) \lstinline{reduce}, which applies a
binary-associative operator $\oplus$ to all elements of the input,
and (iii) \lstinline{scan}, which computes all prefix sums of the
input array elements.  Their types and semantics are shown below:
%
\[ \begin{array}{l}
\map ~ : ~ (\alpha \to \beta) ~ \to ~ \Pi n. [n]\alpha ~ \to [n]\beta \\
\map ~ f ~ [a_1, \ldots, a_n] ~ = ~ [f ~ a_1, \ldots, f ~ a_n]\smallskip\\
\reduce : (\alpha \to \alpha \to \alpha) ~ \to ~ \alpha ~ \to ~ \Pi n.[n]\alpha \to \alpha\\
\reduce ~ \oplus ~ 0_{\oplus} ~ [a_1, \ldots, a_n] ~ = ~ 0_{\oplus} \oplus a_1 \oplus \ldots \oplus a_n\smallskip\\
\scan ~ : ~ (\alpha \to \alpha \to \alpha) ~ \to ~ \alpha ~ \to ~ \Pi n.[n]\alpha ~ \to ~ [n]\alpha\\
\scan ~ \oplus ~ 0_{\oplus} ~ [a_1, \ldots, a_n] ~ = ~ [a_1, \ldots, a_1 \oplus \ldots \oplus a_n]\\
\end{array} \]
\noindent Here $[a_1, \ldots, a_n]$ denotes an array literal,
$[n]\tau$ denotes the type of arrays with $n$ elements of type $\tau$,
and $0_{\oplus}$ denotes the neutral element of the binary associative
operator $\oplus$.  The $\Pi n$ notation indicates where the size $n$
becomes fixed; it indicates, for instance, that we can partially apply
$\map$ to a function and apply the resulting function to arrays of
different sizes.

Tuples and tuple types are denoted by comma-separated values or types,
enclosed in parentheses. We treat the zip/unzip-isomorphic types
$[n](\tau_1, \ldots, \tau_k) \cong [n]\tau_1 , \ldots , [n]\tau_k$ as
interchangeable in any context.  Similarly, we treat the
curry/uncurry-isomorphic types $[m]([n]\tau) \cong [m \times n]\tau$
as interchangeable.  This isomorphic treatment is justified because
both streaming and indexed access to either type can be efficiently
implemented without explicitly applying the isomorphism and
materializing (storing) the result first.
%

%
%\[ \begin{array}{l}
%\fold : ((\alpha \times \alpha \to \alpha) \times \alpha) \to (\beta \to \alpha) \to \Pi n. (\beta^n \to \alpha) \\
%\fold \,(\oplus, 0) \,g\, [b_1, \ldots, b_n] = 0 \oplus g(b_1) \oplus \ldots \oplus g(b_n)
%\end{array} \]
%\noindent In addition, {\tt iota n} creates the iteration-space array
%{\tt[0,$\ldots$,n-1]} and {\tt replicate n a} duplicates the {\tt a}
%argument {\tt n} times.

The SOAC semantics enables powerful rewrite rules.  For example,
mapping an array by a function $f$ followed by mapping the result
with a function $g$ gives the same result as mapping the original
array with the composition of $f$ and $g$:
\[ \begin{array}{l}
(\map ~ f) ~ \circ (\map ~ g) ~ \equiv ~ \map ~ (f ~ \circ ~ g)
\end{array} \]
Applied from left-to-right and from right-to-left this
rule corresponds to producer-consumer (\textit{vertical}) fusion and fission,
respectively.
%
\textit{Horizontal} fusion/fission refers to the case when the two
\lstinline{map}s are independent (i.e., not in any producer-consumer
relation), as in the equation below:
\[ \begin{array}{l}
(\map ~ f ~ x, \map ~ g ~ y) ~ \equiv\map ~ (\lambda(a,b).(f~a,g~b)) ~ (x,y)
\end{array} \]

%More complex rewrite rules are used for fusing chains of \lstinline{map} and
%\lstinline{reduce} operations into a construct named {\tt redomap}~\cite{Futhark:redomap}
%(and similar for {\tt map $\circ$ scan}).

The rest of this section shows how \lstinline{map} and
\lstinline{reduce} are special cases of a more general bulk-parallel
operator named \lstinline{fold}, which (i) can represent (fused)
compositions of \lstinline{map} and \lstinline{reduce} (and
\lstinline{filter}) operators and, as such, (ii) can itself be
decomposed into a \lstinline{map}-\lstinline{reduce} composition.
Similarly, we introduce the parallel operator \lstinline{sFold}, which
generalizes Futhark's streaming operators.


\paragraph{Notation.}

We denote array concatenation by $\#$ and the empty array by $\epsilon$;
$\inj(a)$ is the single-element array containing $a$.
A \emph{partitioning} of an array $v$ is a sequence of arrays $v_1, \ldots, v_k$ such
that $v_1 \# \ldots \# v_k = v$.
%
Given binary operations $f$ and $g$, their \emph{product} $f * g$ is defined by
component-wise application, i.e., $(f, g) (x) = (f~x, g~x)$.
%
%

\paragraph{Parallel Operator fold.}

Many arrays operations are \emph{monoid homomorphisms}, which
conceptually allows for splitting an array into two parts, applying
the operation recursively, and combining the results using an
associative operation $\oplus$.
%
Every monoid homomorphism is uniquely determined by
$(\oplus,0_{\oplus})$ and a function $g$ for mapping singleton
arrays. The combinator $\fold$ thus expresses all such homomorphisms:
\[ \begin{array}{l}
\fold : (\alpha \to \alpha, \alpha) \to (\beta \to \alpha) \to \Pi n. ([n]\beta \to \alpha) \\
\fold \,(\oplus, 0_{\oplus}) \,g\, [b_1, \ldots, b_n] = 0_{\oplus} \oplus (g~b_1) \oplus \ldots \oplus (g~b_n)
\end{array} \]
The $\fold$ combinator can decompose previously seen SOACs:
\[ \begin{array}{l}
\map \, g = \fold\, (\#, \epsilon)\, (\inj \circ g) \\
\reduce\, (\oplus, 0_{\oplus}) = \fold\, (\oplus, 0_{\oplus})\, \Id
\end{array} \]
and $\fold$ can itself be decomposed by the equation
\[
  \fold ~ (\oplus,0_{\oplus}) ~ g ~ = ~ \reduce ~ (\oplus,0_{\oplus}) \circ \map ~ g.
\]

%\paragraph{Streaming combinators.}

\paragraph{Parallel Operator sFold.}
A key aspect of Futhark is to \emph{partition} implementations of $\fold$,
which partitions a vector into \emph{chunks} before applying the operation
on the chunks individually and eventually combining them:
\[ \begin{array}{l}
\streamFold : (\alpha \to \alpha) \to (\Pi m. ([m]\beta \to \alpha)) \to \Pi n. [n]\beta \to \alpha \\
\streamFold \, (\oplus) \, f \, (v_1 \# \ldots \# v_k) = (f~\epsilon) \oplus (f~v_1) \oplus \ldots \oplus (f~v_k) \\[1em]
\end{array} \]
%
Because a vector can have multiple partitions, $\streamFold$ is well-defined---it
gives the same result for all partitions---if and only if $f$ is itself a $\fold$
with $\oplus$ as combining operator.
%and $\oslash$ must be compatible with $\#$:
%$a \oslash (v \# w) = a \oslash v \oslash w$.
%
Futhark assumes such properties to hold; they are not checked at run-time,
but a programmer responsibility.   The streaming combinators
permit Futhark to choose freely \emph{any} suitable partition of the input vector.
Futhark uses specialized versions of $\streamFold$: %and $\StreamSeq$:
\[ \begin{array}{lcl}
\StreamMap\, f & = &  \streamFold \, (\#) \, f \\
\StreamRed\, (\oplus) \, f & = & \streamFold \, ((\oplus) * (\#)) \, f
\end{array} \]

Fusion and fission transformations are based on the universal
properties of $\fold$; for example, horizontal (parallel) fusion is
expressed by the ``banana split theorem'' \cite{mfp91}, read as a
transformation from right to left:
\[
   \begin{array}{l}
  \fold\, ((\oplus, 0_{\oplus}) * (\otimes, 0_{\otimes})) ~ (f, g) \\=(\fold\, (\oplus, 0_{\oplus})\, f, \fold\, (\otimes,
     0_{\otimes})\, g )
   \end{array}
\]

The $\map$-$\map$ rule $\map ~ (g \circ f) = \map ~ g ~ \circ ~ \map ~ f$
is the functorial property of arrays; it is used for fusion from right
to left and eventually, as a fission rule, from left to right as part
of flattening nested parallelism.
%
The flattening rule
$ \map (\map f) \cong \map \,f$
eliminates nested parallel maps by mapping the argument function over
the product index space (an isomorphism modulo the
curry/uncurry isomorphism).
Finally, sequential (de)composition of the discussed SOACs
can be similarly reasoned in terms of more general iterative operators
(known as foldl and sfoldl).


%%% Local Variables:
%%% mode: latex
%%% TeX-master: "thesis"
%%% End:
