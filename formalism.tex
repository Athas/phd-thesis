\newcommand{\czip}{\mathrm{czip}}
\newcommand{\cunzip}{\mathrm{cunzip}}
\newcommand{\cmap}{\mathrm{map}}
\newcommand{\creduce}{\mathrm{reduce}}
\newcommand{\cscan}{\mathrm{scan}}
\newcommand{\cfilter}{\mathrm{scan}}
\newcommand{\inj}{\mathrm{inj}}
\newcommand{\rep}{\mathrm{rep}}
\newcommand{\csFold}{\mathrm{sFold}}
\newcommand{\credomap}{\mathrm{redomap}}
\newcommand{\seg}[1]{\mathbf{#1}}
\newcommand{\tto}{\,\Rightarrow\,}
\newcommand{\iso}{\,\Leftrightarrow\,}
\newcommand{\nat}{\mathbf{N}}
\newcommand{\concat}{\#}
\newcommand{\idd}{\mathrm{id}}
\newcommand{\Id}{\mathrm{id}}
\newcommand{\cfoldl}{\mathrm{foldl}}

\chapter{An Array Calculus}
\label{chap:calculus}

The theoretical foundation of Futhark are the list homomorphisms of
Bird and Meertens~\cite{Bird2}, realised in the form of purely
functional parallel second-order array combinators (SOACs). Their rich
equational theory for semantics-preserving transformations are
employed in Futhark for fusion, streaming, and flattening of
parallelism.  This chapter discusses a simple array combinator
calculus that is used to convey the intuition behind certain useful
transformations on arrays and, more importantly, functions that
operate on arrays.  In \cref{sec:calculus-implementation} we shall
also see that the primitives defined here have an efficient
implementation on the GPU.

The calculus presented here is simpler than the full Futhark
programming language, yet also more flexible.  While the names of the
combinators intentionally resemble those of Futhark (both the source
language and the core language we will see later in the thesis), they
are not exactly identical.

\section{Array Combinator Calculus}
\label{sec:arraycombinators}

\begin{figure}
  \centering
$$
\begin{array}{lrr}
  \alpha,\beta,\tau & ::= & \mbox{\texttt{t}}\ |\ (\tau_1, \ldots, \tau_n)\ |\ [n]\tau \\

  v & ::= & \mbox{\texttt{v}}\ |\ (v_{1}, \ldots, v_{n})\ |\ [v_{1}, \ldots, v_{n}] \\

  \hat{\alpha},\hat{\beta},\hat{\tau} &::=& \tau\ |\ \hat{\alpha}\rightarrow\hat{\beta}\ |\ \Pi n.\hat{\tau} \\

  e &::=& \mbox{\texttt{v}}\ |\ (e_{1}, \ldots, e_{n})\ |\ [e_{1}, \ldots, e_{n}]\ |\ e_1~e_2\ |\ \lambda (x_1,\ldots,x_n).e
\end{array}
$$
  \caption{Syntax of the Array Combinator Calculus.}
  \label{fig:arraycalculus_syntax}
\end{figure}

The array combinator calculus, a modified lambda calculus, describes
terms that operate on array values.  The syntax is summarised on
Figure~\ref{fig:arraycalculus_syntax}.  We make a distinction between
\textit{ground values}
$v$, which cannot contain functionals, and \textit{terms}
$e$, which can.  An array value is written as a comma-separated
sequence of values enclosed in square brackets, as in $[v_{1}, v_{2},
v_{3}]$.  Arrays may contain scalar values written as \texttt{t}
(which are not important for the calculus), other arrays, or tuples of
values.  A tuple is written as a comma-separated sequence of values
enclosed in parentheses.

Array values are classified by array types.  We write $[n]\tau$ to
denote the type of arrays with $n$ elements of type $\tau$.  This
effectively bans irregular arrays, as there is no way to provide an
$n$ such that the value $[[1,2],[3]]$ is classified by a type
$[n]\texttt{i32}$.

The tuple of a tuple is written as the types of its components
separated by commas and enclosed in parentheses.  Primitive
(non-array) types are written as \texttt{t}.  Similarly to other
published array formalisms MOA~\cite{moa}, we treat the
curry/uncurry-isomorphic types $[m]([n]\tau) \cong [m \times n]\tau$
as interchangeable.  This isomorphic treatment is justified because
both streaming and indexed access to either type can be efficiently
implemented without explicitly applying the isomorphism and
materializing (storing) the result first.  This is different from the
Futhark source language, where e.g. $[1][n]\tau$ and $[n][1]\tau$ are
distinct types.  Likewise, a singleton array $[1]\tau$ is distinct
from a scalar of type $\tau$.

Higher-order types $\hat{\tau}$ for classifying functions are a
superset of value types.  Specifically, we do not permit first class
functions in the calculus, although we do permit functional arguments
in function types.  Function types support both ordinary value
abstraction $\hat\alpha\rightarrow\hat\beta$, as well as \textit{size
  abstraction}, written $\Pi n.\hat\tau$.  This allows us to specify a
function that can operate on an array of any size, and returns an
array of that same size, which could have the following type:
\[
  \Pi n.[n]\tau\rightarrow[n]\tau
\]
This size of an array returned by a function must be expressible in
terms of the size of its input.  This restriction could be lifted by
supporting $\exists$-quantification in types, but we omit this for the
array calculus.

\newcommand{\convV}{\mathcal{C}_{\mathcal{V}}}
\newcommand{\convT}{\mathcal{C}_{\mathcal{T}}}

We treat the \texttt{zip}/\texttt{unzip}-isomorphic types
$[n](\tau_1, \ldots, \tau_k) \cong [n]\tau_1 , \ldots , [n]\tau_k$ as
interchangeable in any context.  This is justified as any array of
tuples can be converted into a corresponding tuple of arrays, as shown
on Figure~\ref{fig:calc-tuple-transform}.  The function $\convV(v)$
rewrites a value $v$ such that no arrays of tuples appear.  Likewise,
the function $\convT(\tau)$ rewrites a type.  Intuitively, the
rewriting considers an array of tuples as an augmented array with an
extra dimension, followed by transposing out that extra dimension.
Actually constructing this array would not be well-typed, as all
elements of an array must have the same type, whereas tuples can
contain elements of differing types.

\begin{figure}
  \centering
  \begin{align*}
    &\convV([(v_{(1,1)}, \ldots, v_{(1,m)}),\ldots,(v_{(n,1)}, \ldots, v_{(n,m)})]) =\\
    &\quad (\convV([v_{(1,1)},\ldots,v_{(n,1)}]),\ldots,\convV([v_{(1,m)},\ldots,v_{(n,m)}]))\\
    \\
    &\convT([k](\tau_{1},\ldots,\tau_{n})) =\\
    &\quad(\convT([k]\tau_{1}),\ldots,\convT([k]\tau_{n})) \\
  \end{align*}
  \caption{Converting arrays of tuples to tuples of arrays.  The
    function $\convV$ transforms values, and $\convT$ transforms
    types.  The equalities are applied recursively until a fixed point
    is reached.  We claim the property that if $v : \tau$, then
    $\convV(v) : \convT(\tau)$.}
  \label{fig:calc-tuple-transform}
\end{figure}

\section{Basic SOACs}

We first describe the basic SOACs and later introduce
\textit{streaming combinators}. The basic SOACs include (i) $\cmap$,
which constructs an array by applying its function argument to each
element of the input array, (ii) $\creduce$, which applies a
binary-associative operator $\oplus$ to all elements of the input, and
(iii) $\cscan$, which computes all prefix sums of the input array
elements.  Their types and semantics are shown below:
%
\[ \begin{array}{l}
\cmap ~ : ~ (\alpha \to \beta) ~ \to ~ \Pi n. [n]\alpha ~ \to [n]\beta \\
\cmap ~ f ~ [a_1, \ldots, a_n] ~ = ~ [f ~ a_1, \ldots, f ~ a_n]\smallskip\\
\creduce : (\alpha \to \alpha \to \alpha) ~ \to ~ \alpha ~ \to ~ \Pi n.[n]\alpha \to \alpha\\
\creduce ~ \oplus ~ 0_{\oplus} ~ [a_1, \ldots, a_n] ~ = ~ 0_{\oplus} \oplus a_1 \oplus \ldots \oplus a_n\smallskip\\
\cscan ~ : ~ (\alpha \to \alpha \to \alpha) ~ \to ~ \alpha ~ \to ~ \Pi n.[n]\alpha ~ \to ~ [n]\alpha\\
\cscan ~ \oplus ~ 0_{\oplus} ~ [a_1, \ldots, a_n] ~ = ~ [a_1, \ldots, a_1 \oplus \ldots \oplus a_n]\\
\end{array} \]

\noindent We can view the $\Pi n$ notation as indicating where the
size $n$ becomes fixed; it indicates, for instance, that we can
partially apply $\cmap$ to a function and apply the resulting function
to arrays of different sizes.

%
%\[ \begin{array}{l}
%\credomap : ((\alpha \times \alpha \to \alpha) \times \alpha) \to (\beta \to \alpha) \to \Pi n. (\beta^n \to \alpha) \\
%\credomap \,(\oplus, 0) \,g\, [b_1, \ldots, b_n] = 0 \oplus g(b_1) \oplus \ldots \oplus g(b_n)
%\end{array} \]
%\noindent In addition, {\tt iota n} creates the iteration-space array
%{\tt[0,$\ldots$,n-1]} and {\tt replicate n a} duplicates the {\tt a}
%argument {\tt n} times.

The SOAC semantics enables powerful rewrite rules.  For example,
mapping an array by a function $f$ followed by mapping the result
with a function $g$ gives the same result as mapping the original
array with the composition of $f$ and $g$:
\[ \begin{array}{l}
(\cmap ~ f) ~ \circ (\cmap ~ g) ~ \equiv ~ \cmap ~ (f ~ \circ ~ g)
\end{array} \]
Applied from left-to-right and from right-to-left this
rule corresponds to producer-consumer (\textit{vertical}) fusion and fission,
respectively.
%
\textit{Horizontal} fusion/fission refers to the case when the two
$\cmap{}$s are independent (i.e., not in any producer-consumer
relation), as in the equation below:
\[ \begin{array}{l}
(\cmap ~ f ~ x, \cmap ~ g ~ y) ~ \equiv\cmap ~ (\lambda(a,b).(f~a,g~b)) ~ (x,y)
\end{array} \]

%More complex rewrite rules are used for fusing chains of \lstinline{map} and
%\lstinline{reduce} operations into a construct named {\tt redomap}~\cite{Futhark:redomap}
%(and similar for {\tt map $\circ$ scan}).

The rest of this chapter shows how $\cmap{}$ and $\creduce{}$ are
special cases of a more general bulk-parallel operator named
$\credomap{}$, which (i) can represent (fused) compositions of
$\cmap{}$ and $\creduce{}$ operators and, as such, (ii) can itself be
decomposed into a $\cmap{}$-$\creduce{}$ composition.  Similarly, we
introduce the parallel operator $\csFold{}$, which generalizes
Futhark's streaming operators.

\subsection{Notation}

We denote array concatenation by $\#$ and the empty array by $\epsilon$;
$\inj(a)$ is the single-element array containing $a$.
A \emph{partitioning} of an array $v$ is a sequence of arrays $v_1, \ldots, v_k$ such
that $v_1 \# \ldots \# v_k = v$.
%
Given binary operations $f$ and $g$, their \emph{product} $f * g$ is defined by
component-wise application, i.e., $(f, g) (x) = (f~x, g~x)$.
%
%

\subsection{The Parallel Operator $\credomap$}
\label{sec:formal-redomap}

Many arrays operations are \emph{monoid homomorphisms}, which
conceptually allows for splitting an array into two parts, applying
the operation recursively, and combining the results using an
associative operation $\oplus$.  For example, a $\cmap-\creduce$
composition can be formally transformed, via the list homomorphism
promotion lemma~\cite{BirdListTh}, to an equivalent form:
\[
  \creduce~\oplus~0_{\oplus} \circ \cmap~f \equiv \creduce~\oplus~e \circ \cmap~(\creduce~\oplus~0_{\oplus} \circ \cmap~f) \circ \textrm{split}_{p}
\]
where the original array is partitioned into $p$ chunks.
Operationally, we could imagine that $p$ processors will each perform
the $\cmap-\creduce$ composition sequentially, followed by a parallel
combination of the per-processor results.  Hence, the \textit{inner}
$\cmap-\creduce$ can be written as a left-fold
\[
  \creduce~\oplus~0_{\oplus} \circ \cmap~f \equiv \creduce~\oplus~e \circ \cmap~(\cfoldl~g~0_{\oplus}) \circ \textrm{split}_{p}
\]
for an appropriately defined function $g$.

Thus, every monoid homomorphism is uniquely determined by
$(\oplus,0_{\oplus})$ and a function $g$.  The combinator $\credomap$\footnote{The name is a portmanteau of $\creduce \circ \cmap$.}
thus expresses all such homomorphisms:
\[ \begin{array}{l}
\credomap : (\alpha \to \alpha \to \alpha, \alpha) \to (\beta \to \alpha) \to \Pi n. ([n]\beta \to \alpha) \\
\credomap~(\oplus, 0_{\oplus})~g~[b_1, \ldots, b_n] = 0_{\oplus} \oplus (g~0_{\oplus}~b_1) \oplus \ldots \oplus (g~0_{\oplus}~b_n)
\end{array} \]
The $\credomap$ combinator can decompose previously seen SOACs:
\[ \begin{array}{l}
\cmap \, g = \credomap\, (\#, \epsilon)\, (\inj \circ g) \\
\creduce\, (\oplus, 0_{\oplus}) = \credomap\, (\oplus, 0_{\oplus})\, \Id
\end{array} \]
and $\credomap$ can itself be decomposed by the equation
\[
  \credomap ~ (\oplus,0_{\oplus}) ~ g ~ = ~ \creduce ~ (\oplus,0_{\oplus}) \circ \cmap ~ g.
\]

\subsection{Streaming combinators.}

A key aspect of Futhark is to \emph{partition} implementations of
$\credomap$, such that they partition vectors into \emph{chunks}
before applying the operation on the chunks individually and
eventually combining them:
\[ \begin{array}{l}
\csFold : (\alpha \to \alpha) \to (\Pi m. ([m]\beta \to \alpha)) \to \Pi n. [n]\beta \to \alpha \\
\csFold~(\oplus)~f~(v_1 \# \ldots \# v_k) = (f~\epsilon) \oplus (f~v_1) \oplus \ldots \oplus (f~v_k) \\
\end{array} \]
%
Because a vector can have multiple partitions, $\csFold$ is
well-defined---it gives the same result for all partitions---if and
only if $f$ is equivalent to a $\credomap$ with $\oplus$ as combining
operator.  Futhark assumes such properties to hold; they are not
checked at run-time, but a programmer responsibility.

The streaming combinators permit Futhark to choose freely \emph{any}
suitable partition of the input vector.  In the source language, Futhark uses specialized
versions of $\csFold$:
\[
  \begin{array}{lcl}
    \StreamMap~f & = &  \csFold~(\#)~f \\
    \StreamRed~(\oplus)~f & = & \csFold~((\oplus) * (\#))~f
  \end{array}
\]

Fusion and fission transformations are based on the universal
properties of $\credomap$ (and $\csFold$); for example, horizontal
(parallel) fusion is expressed by the ``banana split theorem''
\cite{mfp91}, read as a transformation from right to left:
\[
   \begin{array}{l}
  \credomap\, ((\oplus, 0_{\oplus}) * (\otimes, 0_{\otimes})) ~ (f, g) =(\credomap\, (\oplus, 0_{\oplus})\, f, \credomap\, (\otimes,
     0_{\otimes})\, g )
   \end{array}
\]

The $\cmap$-$\cmap$ rule
$\cmap ~ (g \circ f) = \cmap ~ g ~ \circ ~ \cmap ~ f$ is the
functorial property of arrays; it is used for fusion from right to
left and eventually, as a fission rule, from left to right as part of
flattening nested parallelism (see \cref{chap:kernel-extraction}).
%
The flattening rule
$ \cmap (\cmap f) \cong \cmap \,f$
eliminates nested parallel maps by mapping the argument function over
the product index space (an isomorphism modulo the
curry/uncurry isomorphism).
Finally, sequential (de)composition of the discussed SOACs
can be similarly reasoned in terms of more general iterative operators
(known as foldl and sfoldl).

%%% Local Variables:
%%% mode: latex
%%% TeX-master: "thesis"
%%% End:
