\chapter{Empirical Validation in Bulk}
\label{chap:empirical-validation}

When a new programming model is proposed, it must be demonstrated that
interesting programs can be written within the model.  While the model
espoused by this thesis---functional array programming---is not new,
our claim that it is a good foundation for obtaining parallel
performance requires evidence.

This chapter contains a discussion of several programs implemented in
Futhark, with a focus on their runtime performance.  While we cannot
claim that the current Futhark compiler embodies the full potential of
functional array programming, the results presented here constitute at
least a lower bound on that potential.

Most of the benchmark programs presented here are based on, and
compared against, hand-written low-level code written in CUDA or
OpenCL.  We have sourced these from the published
Rodinia~\cite{5306797} (version 3.1),
Parboil~\cite{stratton2012parboil} (version 2.5), and
FinPar~\cite{FinPar:TACO} benchmark suites.  The comparison with
hand-written implementations is to show the cost of using a high-level
language.  However, as we shall see, even published code often
contains significant inefficiencies, which occasionally leads to
Futhark outperforming the reference implementations.  A handful of
benchmarks are from Accelerate~\cite{mcdonell2013optimising}, a
Haskell eDSL for parallel programming.  These are included to show the
performance of Futhark compared to an existing mature GPU language.

All programs have been manually ported to Futhark, and compiled and
run with the default settings.  We show how the performance of the
Futhark code compares to the performance of the original reference
implementations.  In some cases, we also discuss the programming
techniques used to obtain efficient Futhark implementations.

\section{N Benchmarks from Rodinia}

The speedup on Backprop seems related to a reduction that Rodinia has
left sequential.  Running time of the training phase is roughly equal
in Rodinia and Futhark ($\sim10~ms$).

Rodinia's implementation of HotSpot uses time
tiling~\cite{HexaTiling}, which seems to pay off on the NVIDIA GPU,
but not on AMD. Futhark's slowdown is due to double buffering via copy
rather than pointer switching, accounting for $30\%$ of runtime.

Our speedup on K-means is due to Rodinia not parallelizing computation
of the new cluster centers, which is semantically a segmented
reduction.

Myocyte's dataset has been expanded because its degree of parallelism
was one (\texttt{workload}=1).  We used Rodinia's CUDA implementation
rather than its OpenCL implementation, as the latter is not fully
parallelised..  We attribute our speedup to automatic coalescing
optimizations, which is tedious to do by hand on such large programs.

NN speedup is due to Rodinia leaving $100$ \kw{reduce} operations for
finding the nearest neighbors sequential on the CPU. Possibly because
the reduce operator is atypical; it computes both the minimal value
and the corresponding index.  Speedup is less impressive on the AMD
GPU, due to higher kernel launch overhead---this benchmark is
dominated by frequent launches of short kernels.

For Pathfinder, Rodinia uses time tiling, which, unlike HotSpot, does
not seem to pay off on the tested hardware.

\section{M Benchmarks from Parboil}

\section{Two Benchmarks from FinPar}
\label{sec:finpar}

OptionPricing from FinPar is essentially a
\lstinline{map}-\lstinline{reduce}-composition.  The benchmark primarily
measures how well the Futhark compiler sequentialises excess
parallelism inside the complex \lstinline{map} function.

LocVolCalib from FinPar is an outer \lstinline{map} containing a
sequential \lstinline{for}-loop, which itself contains several more
\lstinline{map}s.  Exploiting all parallelism requires the compiler to
interchange the outer \lstinline{map} and the sequential loop.  The
slowdown on the AMD GPU is due to transpositions, inserted to fix
coalescing, being relatively slower than on the NVIDIA GPU.


\section{L Benchmarks from Accelerate}
\label{sec:accelerate}

The four benchmarks Crystal, Fluid, Mandelbrot, and N-body are from
Accelerate.  The N-body simulation comprises a width-$N$ map where
each element performs a fold over each of the $N$ bodies.

\section{Impact of Optimisations}

Impact was measured by turning individual optimisations off and
re-running benchmarks on the NVIDIA GPU.  We report only where
the impact is non-neglible.

Fusion has an impact on K-means ($\times1.42$), LavaMD
($\times4.55$), Myocyte ($\times1.66$), SRAD ($\times1.21$), Crystal
($\times10.1$), and LocVolCalib ($\times9.4$).  Without fusion,
OptionPricing, N-body, and MRI-Q fail due to increased
storage requirements.

In the absence of in-place updates, we would have to implement
K-means as on Figure~\ref{fig:parallel-counts}---the resulting
program is slower by $\times8.3$.  Likewise, LocVolCalib
would have to implement its central \lstinline{tridag} procedure via a
less efficient \lstinline{scan}-\lstinline{map} composition, causing a
$\times1.7$ slowdown.  OptionPricing uses an inherently sequential
Brownian Bridge computation that is not expressible without in-place
updates.

The coalescing transformation has an impact on K-means
($\times9.26$), Myocyte ($\times4.2$), OptionPricing ($\times 8.79$),
and LocVolCalib ($\times 8.4$).
%
Loop tiling has an impact on LavaMD ($\times 1.35$), MRI-Q
($\times1.33$), and N-body ($\times 2.29$).


%%% Local Variables:
%%% mode: latex
%%% TeX-master: "thesis"
%%% End:
