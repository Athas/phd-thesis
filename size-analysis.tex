\chapter{Size Analysis}
\label{chap:size-analysis}

A great many optimisations and safety checks in Futhark depend on how
the shape of two arrays relate to each other, and at which point the
shape of an array can be known.  Especially the latter is important.
Nested parallelism supports construction arrays whose \textit{value}
is dependent on some outer parallel construct.  However, for
\textit{regular} nested parallelism, the shapes of those arrays can be
computed invariant to all parallel loops.  It is crucial for efficient
execution that we can hoist the computation of such sizes out of
parallel loops.  This requires us to reify the notion of an array
shape, and the computation of that shape, in the IR.  In the Futhark
compiler, we treat size computations like any other expression, which
allows us to use our general compiler optimisation repertoire to
optimise and simplify the computation of sizes.  We maintain the
invariant that for \textit{any} array-typed variable in scope, each
dimension of that array corresponds to some \lstinline{i32}-typed
variable also in scope.  This chapter discusses how we augment the IR
to provide this property, and how we move from the unsized IR to the
sized IR (analogous to ``typed'' versus ``untyped'').

\section{A Sized IR}

One important observation of the IR presented in the previous chapter
is that some operator-semantics invariants, related to the array
regularity, are guaranteed to hold by construction, but several other
invariants are only ``assumed'', that is, they have not been verified
(made explicit) in the IR:
\begin{itemize}
\item \lstinline{iota} and \lstinline{replicate} assume a non-negative
  first argument, and the size of the resulting array is exactly the
  value of the first argument.
\item \lstinline{map} is guaranteed to receive arguments of identical
  outermost size, which also matches the outermost size of all result
  arrays\footnote{ In the user language \lstinline{zip} accepts an
    arbitrary number of array arguments that are required to have the
    same outermost size.}.  However, \lstinline{map} assumes that its
  function argument produces arrays of identical shape for each
  element of the input array.
\item \lstinline{filter} receives and produces arguments and results
  of identical outermost size, respectively (and the outermost size of
  the argument is not smaller than the one of the result).
\item \lstinline{reduce} and \lstinline{scan} receive arguments of
  identical outermost size, and \lstinline{scan} results have
  outermost size equal to that of the input.  The semantics for
  \lstinline{reduce} and \lstinline{scan} assumes that the
  (corresponding two) arguments and result of the binary associative
  operator have identical shapes.
\end{itemize}

\begin{figure}
\begin{tabular}{lrlr}
  $\tau$ & $::=$ & \mbox{\texttt{t} ~ $|$ ~ \texttt{[$d$]$\tau$}} & \mbox{(Scalar/array type)} \\
  $\ft$ & $::=$ & $(x_{1}: \tau_1) \rightarrow \cdots \rightarrow (x_{n}: \tau_{n}) \rightarrow \exists \bar{d_2}.\tty$ & (Sizes of results $\in \bar{d_2}$)\\
  $\etau$ & $::=$ & $\alpha~|~(x_{1}: \tau_{1}, \ldots, x_{n}: \tau_{n})~|~(x: \etau) \rightarrow \etau$ \\
  $\opty$ & $::=$ & $\forall\bar{\alpha}.\etau$ \\
  fun & ::= & \mbox{\Fun{$f$}{$\hat{p}$}{$\overline{d}.\utau$}{$b$}} ~~& ~~\mbox{(Named function)} \\
\end{tabular}
\caption{Types with embedded size information and named parameters.
  The remaining syntax definitions remain unchanged.}
\label{fig:sizeTypes}
\end{figure}

\begin{figure}
\begin{tabular}{lcl}
  \emph{op} & & \textrm{TySch}(\emph{op}) \\\hline
  {\lstinline!iota!} & : & $(d: \texttt{i32}) \rightarrow [d]\texttt{i32}$ \\
  {\lstinline!replicate!} & : & $\forall\alpha.(d: \texttt{i32}) \rightarrow \alpha \rightarrow [d]\alpha$ \\
  \lstinline[mathescape]!reshape! & : & $\forall\nseq{d}{m}\alpha.(x_{1}: \texttt{i32}, \ldots, x_{n}: \texttt{i32})$ \\
  & & ~~~~~~~~~~~~ $\rightarrow[d_{1}]\cdots[d_{m}]\alpha\rightarrow[x_{1}]\cdots[x_{n}]\alpha$ \\
  \lstinline[mathescape]!rearrange ($c_{1}, \ldots, c_{n}$)! & : & $\forall\nseq{d}{n}\alpha.[d_{1}]\cdots[d_{n}]\alpha\rightarrow[d_{p(1)}]\cdots[d_{p(n)}]\alpha$ \\
            & & ~~ \text{where $p(i)$ is the result of applying the}\\
            & & ~~ \text{permutation induced by $c_{1}, \ldots, c_{n}$}. \\
{\lstinline!map!} & : & $\forall d\ \bar{\alpha}^{(n)}\bar{\beta}^{(m)}.(\alpha_1 \rightarrow \ldots \rightarrow \alpha_n \rightarrow (\bar{\beta}^{(m)}))$\\
          & & ~~~~~~~~~~~~~~~~~~~~ $\rightarrow [s]\alpha_1\rightarrow\cdots\rightarrow[s]\alpha_n$\\
          & & ~~~~~~~~~~~~~~~~~~~~ $\rightarrow ([s]\beta_1,\ldots,[s]\beta_m)$\\
  {\lstinline!reduce!} & : & $\forall \ d \ \bar{\alpha}^{(n)}.(\alpha_1 \rightarrow \ldots \rightarrow \alpha_n \rightarrow \alpha_1 \rightarrow \ldots \rightarrow \alpha_n \rightarrow (\bar{\beta}^{(n)}))$ \\
          & & ~~~~~~~~~~~~~~ $\rightarrow (\alpha_1, \ldots, \alpha_n)$\\
          & & ~~~~~~~~~~~~~~ $\rightarrow [d]\alpha_1 \rightarrow \ldots \rightarrow [d]\alpha_n$\\
          & & ~~~~~~~~~~~~~~ $\rightarrow (\alpha_1,\ldots,\alpha_m)$ \\
  {\lstinline!scan!} & : & $\forall \ d \ \bar{\alpha}^{(n)}.(\alpha_1 \rightarrow \ldots \rightarrow \alpha_n \rightarrow \alpha_1 \rightarrow \ldots \rightarrow \alpha_n \rightarrow (\bar{\beta}^{(n)}))$ \\
          & & ~~~~~~~~~~~~~~ $\rightarrow (\alpha_1, \ldots, \alpha_n)$\\
          & & ~~~~~~~~~~~~~~ $\rightarrow [d]\alpha_1 \rightarrow \ldots \rightarrow [d]\alpha_n$\\
          & & ~~~~~~~~~~~~~~ $\rightarrow ([d]\alpha_1,\ldots,[d]\alpha_m)$ \\
{\lstinline!filter!} & : & $\forall \ d_1 \ \bar{\alpha}^{(n)}.(\alpha_1 \rightarrow \ldots \rightarrow \alpha_n \rightarrow \texttt{bool})$ \\
          & & ~~~~~~~~~~~~~~ $\rightarrow [d_{1}]\alpha_1 \rightarrow \ldots \rightarrow [d_{1}]\alpha_n$ \\
          & & ~~~~~~~~~~~~~~ $\rightarrow \exists d_{2}.([d_{2}]\alpha_1,\ldots,[d_{2}]\alpha_n)$ \\
\end{tabular}
\caption{Dependent-size types for various SOACs.}
\label{fig:soacSizeType}
\end{figure}

Figure~\ref{fig:sizeTypes} shows an extended type system in which (i)
sizes are encoded in each array type, that is, $[d]\tau$ represents
the type of an array in which the outermost dimension has size $d$,
and in which (ii) function/lambda types use universal quantifiers for
the sizes of the array parameters ($\forall s_1$), and existential
quantifiers for the sizes of the result arrays ($\exists s_2$).
Function types now also contain \textit{named} parameters, supporting
a simple variant of dependent types.  For function parameters where
the name is irrelevant, we shall elide the name and use the same
notation as previously.
%
Figure~\ref{fig:soacSizeType} also shows that this extension allows to
encode most of the afore-mentioned invariants into size-dependent
types.  The requirement for non-negative input to \lstinline{iota} and
\lstinline{replicate} remains a dynamic property.  We see that most
parameters remain unnamed, but are crucially used to encode the shape
properties of \lstinline{iota} and \lstinline{replicate}.

The type of \lstinline{map} is interesting because the result array
types do not follow immediately from the input array types.  Instead,
it is expected that the functional argument declares the result type
(including sizes) in advance.  Operationally, we can see this as being
able to ``pre-allocate'' space for the result.  However, the return
size cannot in general be known in advance without evaluating the
function.
% 
The following demonstrates, by example, the code transformation that
(i) makes explicit in the code the shape-dependent types and verifies
the assumed invariants and (ii) optimizes away in many cases the
existential types.

\section{Size Analysis by Example}
\label{subsec:size-analysis-intuition}


\begin{figure}
\begin{lstlisting}
let concat (xs: []f64) (ys: []f64): []f64 =
  let a = size 0 xs
  let b = size 0 ys
  let c = a+b
  let (is: []i32) = iota c
  let (zs: []f64) = map (\i -> if i < a then xs[i] else ys[i+b])
                        is
  in zs

let f (vss: [][]f64): []f64 =
  let (ys: []f64) (zs: []f64) =
    map (\(vs: []f64) ->
          let ys = reduce (+) 0.0 vs
          let zs = reduce (*) 1.0 vs
          in (ys, zs))
        vss
  let (rs: []f64) = concat ys zs
  in rs

let main (vsss: [][][]f64): [][]f64 =
  let (rss: [][]f64) =
    map (\(vss: [][]f64): []f64 ->
           let rs = f vss
           in rs) vs
  in rss
\end{lstlisting}

\caption{Running example: Program in un-sized IR.} 
\label{fig:RunEgSrc}
\end{figure}

The program in Figure~\ref{fig:RunEgSrc} receives as input a
three-dimensional array \texttt{A}, and produces a two-dimensional
array \texttt{B}, by mapping the elements of the outermost dimension
of \texttt{A} by function \texttt{f}, such that each of the rows of
\texttt{B} is the catenation of the arrays produced by reducing the
innermost dimension of \texttt{A} by addition and multiplication,
respectively. The Futhark source language is very close to the
un-sized IR in Figure~\ref{fig:RunEgSrc}, except that types are
required only in function declarations, and expressions need not be in
A-normal form.

The parameters/result of \texttt{main} are treated in adhoc fashion
(read/written from/to a file).
Note that the source IR is size-agnostic (typed): 
for example, \texttt{B} has been inferred to be a two-dimensional array of 
reals \texttt{--} as a result of a \texttt{map} with a function of signature
\texttt{[[real]] $\rightarrow$ [real]} \texttt{--} but its shape is not known yet. 

\begin{figure}
\begin{lstlisting}
let concat @(n: i32) (m: i32)@ (xs: [@n@]f64) (ys: [@m@]f64): @d.@[@d@]f64 =
  let a = @n@
  let b = @m@
  let c = a+b
  let (is: [@c@]i32) = iota c
  let (zs: [@c@]f64) =
   map (\i -> if i < n then xs[i] else ys[i+a]) is
  in zs

let f @(m: i32) (k: i32)@ (vss: [@m@][@k@]f64): @d@.[@d@]f64 =
  let (ys: [@m@]f64) (zs: [@m@]f64) =
    map (\(vs: [@k@]f64) ->
          let ys = reduce (+) 0.0 vs
          let zs = reduce (*) 1.0 vs
          in (ys, zs))
        vss
  let @(l: i32)@ (rss: [@l@]f64) = concat @m m@ ys zs
  in rs

let main @(n: i32) (m: i32) (k: i32)@
         (vsss: [@n@][@m@][@k@]f64): @d1 d2.@[@d@][@d1@][@d2@]f64 =
@  let l = if n != 0@
@          then let (d: i32) (ws: [d]f64) = f n m vsss[0]@
@               in d@
@          else 0@
  let (rss: [@n@][@d@]f64) =
    map (\(vss: [@m@][@k@]f64): [@l@]f64 ->
           let @(d: i32)@ @(w: [d]f64)@ = f @m k@ vss
@           let vs = reshape l w@
           in vs)
        vsss
  in rss
\end{lstlisting}

  \caption{Running example:
    $\exists$-quantified target IR.  Changes compared to
    Figure~\ref{fig:RunEgSrc} highlighted in red.}
\label{fig:RunEgTgt}
\end{figure}

The first stage, demonstrated in Figure~\ref{fig:RunEgTgt}, transforms
the program into an unoptimised version in which (i) all arrays have
shape-dependent types, which may be existentially quantified, and (ii)
all ``assumed'' invariants are explicitly checked.
%
This is achieved by:
\begin{itemize}
\item Extending the function signatures to encompass also the shape
  information for each array argument.  For example, \lstinline{f}
  takes additional parameters \lstinline{m} and \lstinline{k} that
  specify the shape of array argument \lstinline{vss},

\item Representing function's array results via
  existentially-quantified shape-dependent types.  For example, the
  return type of of \lstinline{f} is specified as
  \lstinline{d.[d]f64}, indicating an existential size \lstinline{d}.

\item Modifying \lstinline{let} patterns to also explicitly bind any
  existential sizes returned by the corresponding expression.  For
  example, the binding of the result of \lstinline{concat} now also
  includes a variable \lstinline{l}, representing the result size.

\item For the \lstinline{map} in \lstinline{main}, we need to make a
  ``guess'' at the size of the array being returned, which we store as
  the variable \lstinline{l}..  This guess is made by applying the
  anonymous function to \lstinline{vsss[0]}, which produces both a
  size and an array, from which we use just the size.  If
  \lstinline{vsss} is empty (that is, if \lstinline{n} is zero), the
  guess is zero.

  Since \lstinline{f} returns an existential result, and the lambda
  \textit{must} return an array of exactly type \lstinline{[l]f64}, we
  use a \lstinline{reshape} to obtain this desired type.  Since the
  \lstinline{reshape} fails if \lstinline{d != n}, this effectively
  ensures that the \lstinline{map} produces a regular array.

\item Since all arrays in scope also have variables in scope for
  describing their size, replace all uses of \lstinline{size} with
  references to those varuables.
\end{itemize}

It is important to note that this transformation preserves
asymptotically the number of operations of the original program.
However, it performs a significant amount of redundant computation.
To compute \lstinline{l}, we compute the entire result, only to throw
most of it away.  General-purpose optimisation techniques can be
employed to eliminate the overhead.  On
Figure~\ref{fig:SimplifyFShape} we see the result of inlining all
functions, followed by straightforward simplification, dead code
removal, and hoisting of the computation of \lstinline{c}.  The result
is that all arrays constructed insice the \lstinline{map}s have a size
that can be computed before the \lstinline{map}s are entered.  From an
operational perspective, this lets us pre-allocate memory before
executing the \lstinline{map}s on a GPU.  This is essential for GPU
execution because dynamic allocation and assertions are typically not
well suited for accelerators, hence the shapes of the result and of
various intermediate arrays need to be computed (or at least
overestimated) and verified before the kernel is run.

\begin{figure}
\begin{lstlisting}
let main (n: i32) (m: i32) (k: i32)
         (vsss: [n][m][k]f64): d1 d2.[d][d1][d2]f64 =
  let l = if n != 0 then m+m else 0
  let c = m+m
  let (rss: [n][d]f64) =
    map (\(vss: [m][k]f64): [l]f64 ->
          let (ys: [m]f64) (zs: [m]f64) =
            map (\(vs: [k]f64) ->
                  let ys = reduce (+) 0.0 vs
                  let zs = reduce (*) 1.0 vs
                  in (ys, zs))
                vss
          let (is: [c]i32) = iota c
          let (zs: [c]f64) =
            map (\i -> if i < n then xs[i] else ys[i+a]) is
          let rs = reshape l zs
          in rs)
        vsss
  in rss
\end{lstlisting}

  \caption{After inlining all functions and performing simple
    inlining, dead-code elimination, and simplification---no
    existential quantification left.}
\label{fig:SimplifyFShape}
\end{figure}

However, there is still a problem with the current form of the code.
The issue is that the compiler cannot statically see that
\lstinline{l==m}, and thus has to maintain the \lstinline{reshape} and
perform a dynamic safety check at run-time.  This is because the
computation of \lstinline{l} is hidden behind a branch.  The branch
was conservatively inserted because we could not be sure that the
value of \lstinline{vsss[0]} would not be used for computing the size
(size analysis is intraprocedural, and so we have no insight in the
definition of \lstinline{f}), but now it is a hindrance to further
simplification.  There are at least two possible solutions, both of
which are used by the present Futhark compiler:

\begin{enumerate}
\item Give the programmer the ability to annotate the original lambda
  (in the source language) with the return type, \textit{including}
  expected size.  This effectively lets the programmer make the guess
  for us, with no branch required.  The result is still checked by a
  \lstinline{reshape}, but in most cases the guess will be statically
  the same as the computed size, and the \lstinline{reshape} can thus
  be simplified away.
\item Somehow ``mark'' the branch as being a size computation.  Then,
  after inlining and simplification, we can recognise such branches,
  and simplify them to their ``true'' branch, if that branch contains
  only ``safe'' expressions, where a safe expression is one whose
  evaluation can never fail.  We have to wait until after inlining and
  simplification, as a function call can never be considered safe.

  This solution has the downside that it may affect whether an inner
  size of an empty array is zero or nonzero.
\end{enumerate}

In practise, the first solution is preferable in the vast majority of
cases, as it also serves as useful documentation of programmer intent
in the source program.

A third solution is to factor out the ``checking'' part of the
\lstinline{reshape} operation.  This approach is sketched on
Figure~\ref{fig:SimplifyFShapeCert}.  Here, we use a hypothetical
\lstinline{assert} expression for computing a ``certificate'', on
which the \lstinline{reshape} expression itself is predicated.  To
enable the check to be hoisted safely out of the outer
\lstinline{map}, the condition also succeeds if the outer map contains
no iterations (\lstinline{n == 0}).  The Futhark compiler currently
makes only limited use of this technique, as the static equivalence of
sizes is a more powerful enabler of other optimisations.

One may observe that in the resulting code, the shape and regularity
of \texttt{rss} are computed and verified before the definition of
\texttt{rss}, respectively, and, most importantly, that the size
computation and assumed-invariant verification introduce negligible
overhead, i.e., $O(1)$ number of operations.

\begin{figure}
\begin{lstlisting}
let main (n: i32) (m: i32) (k: i32)
         (vsss: [n][m][k]f64): d1 d2.[d][d1][d2]f64 =
  let l = if n != 0 then m+m else 0
  let c = m+m
@  let cert = assert(n == 0 || l==c)@
  let (rss: [n][d]f64) =
    map (\(vss: [m][k]f64): [l]f64 ->
          let (ys: [m]f64) (zs: [m]f64) =
            map (\(vs: [k]f64) ->
                  let ys = reduce (+) 0.0 vs
                  let zs = reduce (*) 1.0 vs
                  in (ys, zs))
                vss
          let (is: [c]i32) = iota c
          let (zs: [c]f64) =
            map (\i -> if i < n then xs[i] else ys[i+a]) is
          let rs = reshape@<cert>@ l zs
          in rs)
        vsss
  in rss
\end{lstlisting}

  \caption{Separating the size-checking of a \lstinline{reshape} from
    the \lstinline{reshape} itself.}
\label{fig:SimplifyFShapeCert}
\end{figure}

Note that the code shown on Figure~\ref{fig:RunEgTgt} is the only
\textit{required} step we have to perform.  Subsequent optimisation to
eliminate existential quantification could be done in any way that is
found desirable, perhaps via interprocedural analysis or slicing.
Previously, we experienced with a technique based on \textit{slicing},
where a function \lstinline{g} is divided into two functions
\lstinline{g_shape} and \lstinline{g_value}, the first of which
computes the sizes of all (top-level) arrays in the latter, including
the result.  An example is shown on Figure~\ref{fig:FShapeSlice},
which contains a portion of the running example.  The
\lstinline{concat} function has been split into
\lstinline{concat_shape} and \lstinline{concat_value}. The call to
\lstinline{concat} has been likewise split.  The Futhark compiler
currently does not use this approach.  Partly because of the risk for
asymptotic slowdown in the presence of recursion (which was supported
at the time), and partly because merely inlining plus simplification
is easier to implement, and performed equally well.

\begin{figure}

\begin{lstlisting}
let concat_shape (n: i32) (m: i32) (xs: [n]f64) (ys: [m]f64): i32 =
  n+m

let concat_value (n: i32) (m: i32) (c: i32) (xs: [n]f64) (ys: [m]f64): [c]f64 =
  let (is: [c]i32) = iota c
  let (zs: [c]f64) =
    map (\i -> if i < n then xs[i] else ys[i+n]) is
  in zs

let f (m: i32) (k: i32) (vss: [m][k]f64): d.[d]f64 =
  let (ys: [m]f64) (zs: [m]f64) =
    map (\(vs: [k]f64) ->
          let ys = reduce (+) 0.0 vs
          let zs = reduce (*) 1.0 vs
          in (ys, zs))
        vss
  let (l: i32) = concat_shape m m c ys zs
  let (rss: [l]f64) = concat_value m m c ys zs
  in rs
\end{lstlisting}

  \caption{An example of applying the slicing approach to \lstinline{concat}.}
  \label{fig:FShapeSlice}
\end{figure}

%%% Local Variables:
%%% mode: latex
%%% TeX-master: "thesis"
%%% End:
