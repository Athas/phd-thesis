\chapter{Size Analysis}
\label{chap:size-analysis}

A great many optimisations and safety checks in Futhark depend on how
the shape of two arrays relate to each other, or at which point the
shape of an array can be known (especially if that point is much
earlier than the point at which the \textit{values} of the array can
be known).  Especially the latter is important for the moderate
flattening transformation (\cref{chap:kernel-extraction}) that
constitutes one of the main contributions of the thesis.  Nested
parallelism supports the construction of arrays whose values are
dependent on some outer parallel construct.  However, for
\textit{regular} nested parallelism, the shapes of those arrays can be
computed invariant to all parallel loops.  It is crucial for efficient
execution that we can hoist the computation of such sizes out of
parallel loops.  This requires us to reify the notion of an array
shape, and the computation of that shape, in the IR.  In the Futhark
compiler, we treat size computations like any other expression, which
allows us to use our general compiler optimisation repertoire to
optimise and simplify the computation of sizes.

The main contribution of this chapter is an IR design that maintains
the invariant that for \textit{any} array-typed variable in scope,
each dimension of that array corresponds to some \lstinline{i32}-typed
variable also in scope.  For expressions where the shape of the result
cannot be computed in advance (consider a \kw{filter} or a function
call), we use a lightweight mechanism based on \textit{existential
  types}.  This chapter also discusses how we move from the unsized IR
to the sized IR (analogous to ``typed'' versus ``untyped''), and how
most size calculations can subsequently be optimised away.  An
important technique is \textit{function slicing}, which we use to
derive functions that precompute the size of values returned by a
function.

\section{A Sized IR}

One important observation of the IR presented in the previous chapter
is that some operator-semantics invariants, related to the array
regularity, are guaranteed to hold by construction, but several other
invariants are only ``assumed'', that is, they have not been verified
(made explicit) in the IR:
\begin{itemize}
\item \lstinline{iota} and \lstinline{replicate} assume a non-negative
  first argument, and the size of the resulting array is exactly the
  value of the first argument.
\item \lstinline{map} is guaranteed to receive arguments of identical
  outermost size, which also matches the outermost size of all result
  arrays\footnote{ In the user language \lstinline{zip} accepts an
    arbitrary number of array arguments that are required to have the
    same outermost size.}.  However, \lstinline{map} assumes that its
  function argument produces arrays of identical shape for each
  element of the input array.
\item \lstinline{filter} receives and produces arguments and results
  of identical outermost size, respectively (and the outermost size of
  the argument is not smaller than the one of the result).
\item \lstinline{reduce} and \lstinline{scan} receive arguments of
  identical outermost size, and \lstinline{scan} results have
  outermost size equal to that of the input.  The semantics for
  \lstinline{reduce} and \lstinline{scan} assumes that the two
  arguments and result of the binary associative operator have
  identical shapes.
\end{itemize}

\begin{figure}
\begin{tabular}{lrlr}
  $\tau$ & $::=$ & \mbox{\texttt{t} ~ $|$ ~ \texttt{[$d$]$\tau$}} & \mbox{(Scalar/array type)} \\
  $\ft$ & $::=$ & $(x_{1}: \tau_1) \rightarrow \cdots \rightarrow (x_{n}: \tau_{n}) \rightarrow \exists \bar{d}.\tty$ & (Sizes of results $\in \bar{d}$)\\
  $\etau$ & $::=$ & $\alpha~|~(x_{1}: \tau_{1}, \ldots, x_{n}: \tau_{n})~|~(x: \etau) \rightarrow \exists \bar{d}.\etau$ & ~~\mbox{(Sizes of results $\in \bar{d}$)} \\
  $\opty$ & $::=$ & $\forall\bar{\alpha}.\ft$ \\
  fun & ::= & \mbox{\Fun{$f$}{$\hat{p}$}{$\overline{d}.\utau$}{$e$}} ~~& ~~\mbox{(Sizes of results $\in \bar{d}$)} \\
\end{tabular}
\caption{Types with embedded size information and named parameters.
  The remaining syntax definitions remain unchanged, except that they
  refer to the new definitions above.}
\label{fig:sizeTypes}
\end{figure}

\begin{figure}
\begin{tabular}{lcl}
  \emph{op} & & \textrm{TySch}(\emph{op}) \\\hline
  {\lstinline!iota!} & : & $(d: \texttt{i32}) \rightarrow [d]\texttt{i32}$ \\
  {\lstinline!replicate!} & : & $\forall\alpha.(d: \texttt{i32}) \rightarrow \alpha \rightarrow [d]\alpha$ \\
  \lstinline[mathescape]!reshape! & : & $\forall\nseq{d}{m}\alpha.(x_{1}: \texttt{i32}, \ldots, x_{n}: \texttt{i32})$ \\
            & & ~~~~~~~~~~~~ $\rightarrow[d_{1}]\cdots[d_{m}]\alpha\rightarrow[x_{1}]\cdots[x_{n}]\alpha$ \\
  \lstinline[mathescape]!rearrange ($\nseq{c}{n}$)! & : & $\forall\nseq{d}{n}\alpha.[d_{1}]\cdots[d_{n}]\alpha\rightarrow[d_{p(1)}]\cdots[d_{p(n)}]\alpha$ \\
            & & ~~ \text{where $p(i)$ is the result of applying the}\\
            & & ~~ \text{permutation induced by $c_{1}, \ldots, c_{n}$}. \\
  {\lstinline!map!} & : & $\forall d\ \bar{\alpha}^{(n)}\bar{\beta}^{(m)}.$\\
            & & $~(\nseq{\alpha}{n} \rightarrow (\bar{\beta}^{(m)})) \rightarrow \nseq{[s]\alpha_i}{n} \rightarrow (\nseq{[s]\beta_i}{m})$\\
  {\lstinline!scatter!} & : & $\forall d \nseq{x}{m} \nseq{\alpha}{n}.$\\
            & & $~(\nseq{[x_{i}]\alpha_{i}}{n}) \rightarrow (\nseq{[d]\texttt{i32}}{n}) \rightarrow (\nseq{[d]\alpha_i}{n}) \rightarrow (\nseq{[x_{i}]\beta_i}{m})$\\
  {\lstinline!reduce!} & : & $\forall d \ \bar{\alpha}^{(n)}.$\\
  & & $~(\nseq{\alpha}{n} \rightarrow \nseq{\alpha}{n} \rightarrow (\bar{\beta}^{(n)})) \rightarrow (\nseq{\alpha}{n}) \rightarrow \nseq{[d]\alpha_i}{n} \rightarrow (\nseq{\alpha}{n})$ \\
  {\lstinline!scan!} & : & $\forall d \ \bar{\alpha}^{(n)}.$\\
  & & $(\nseq{\alpha}{n} \rightarrow \nseq{\alpha}{n} \rightarrow (\bar{\beta}^{(n)})) \rightarrow (\nseq{\alpha}{n}) \rightarrow \nseq{[d]\alpha_i}{n} \rightarrow (\nseq{[d]\alpha_i}{n})$ \\
  {\lstinline!filter!} & : & $\forall d_1 \ \bar{\alpha}^{(n)}.$\\
  & & $~(\nseq{\alpha}{n} \rightarrow \texttt{bool}) \rightarrow \nseq{[d_{1}]\alpha_i}{n} \rightarrow \exists d_{2}.(\nseq{[d_{2}]\alpha_i}{n})$ \\
  \kw{stream\_map} & : & $\forall d~\nseq{\alpha}{n}~\nseq{\beta}{m}.$\\
  & & $~((x: \texttt{i32})\rightarrow\nseq{[x]\alpha_{i}}{n} \rightarrow (\nseq{[x]\beta_{i}}{m})) \rightarrow \nseq{[d]\alpha_i}{n} \rightarrow (\nseq{[d]\beta}{m})$ \\
  \kw{stream\_red} & : & $\forall d~\nseq{\alpha}{n}\nseq{\beta}{m}.(\nseq{\beta}{m} \rightarrow \nseq{\beta}{m} \rightarrow (\nseq{\beta}{m})) $ \\
  & & $~~~\rightarrow ((x: \texttt{i32})\rightarrow\nseq{[x]\alpha_{i}}{n} \rightarrow (\nseq{\beta_{i}}{m}))$ \\
  & & $~~~\rightarrow \nseq{[d]\alpha_i}{n} \rightarrow (\nseq{\beta}{m})$ \\

\end{tabular}
\caption{Dependent-size types for various SOACs.}
\label{fig:soacSizeType}
\end{figure}

\Cref{fig:sizeTypes} shows an extended type system in which (i)
sizes are encoded in each array type, that is, $[d]\tau$ represents
the type of an array in which the outermost dimension has size $d$,
and in which (ii) function/lambda types use universal quantifiers for
the sizes of the array parameters ($\forall s_1$), and existential
quantifiers for the sizes of the result arrays ($\exists s_2$).
Function types now also contain \textit{named} parameters, supporting
a simple variant of dependent types.  For function parameters where
the name is irrelevant, we shall elide the name and use the same
notation as previously.
%
\Cref{fig:soacSizeType} also shows that this extension allows to
encode most of the afore-mentioned invariants into size-dependent
types.  The requirement for non-negative input to \lstinline{iota} and
\lstinline{replicate} remains a dynamic property.  We see that most
parameters remain unnamed, but are crucially used to encode the shape
properties of \lstinline{iota} and \lstinline{replicate}.

The type of \lstinline{map} is interesting because the result array
types do not follow immediately from the input array types.  Instead,
it is expected that the functional argument declares the result type
(including sizes) in advance.  Operationally, we can see this as being
able to ``pre-allocate'' space for the result.  However, the return
size cannot in general be known in advance without evaluating the
function.

\subsection{New Type Rules}

The addition of size-dependent types requires an extension of the
typing rules.  The main problem is the handling of an existential
context in the return type of an expression.  We extent the type
judgment for expressions such that it now returns an a type with an
existentially quantified part (which may be empty).  The most
interesting rules are shown on \cref{fig:sizeTypeRules} and discussed
below.

The rule for \kw{let}-bindings is interesting in that it may
``increase'' the existensiality of a type.  As an example, consider an
expression \lstinline{replicate x v}.  Supposing $\texttt{v}: \tau$,
this expression has type $[x]\tau$ %
while \mbox{\lstinline{let x = y in replicate x v}} has type
$\exists d.[d]\tau$.  The specific names used in the existential
quantification do not matter.

It is also in the rule for \kw{let}-bindings that we bind the
existential sizes to concrete variables.  For example, if the return
type of the function contains $l$ existential sizes, then we require
that the pattern begins with $l$ names of type \texttt{i32}.
Operationally, these will be bound to the actual sizes of the value
returned by the function.  For the type rule, there must be some
substitution $S_{r}$ such that, when $S_{r}$ is applied to the return
type of the function, the resulting type matches the types bound by
the non-existential part of the pattern.  For example, given a
function
\[
  \texttt{f}: (\texttt{x}: \texttt{i32}) \rightarrow (\texttt{y}: \tau) \rightarrow \exists \texttt{n}.[\texttt{x}][\texttt{n}]\tau
\]
we can conclude that the \kw{let}-binding
\[
\Let{(\texttt{m}: \texttt{i32}, \texttt{v}: [\texttt{a}][\texttt{m}]\tau)}{\texttt{f}~\texttt{a}~\texttt{b}}{e}
\]
is well-typed (assuming $\texttt{a}: \texttt{i32}$ and $e$ is
well-typed), but
\[
\Let{(\texttt{m}: \texttt{i32}, \texttt{v}: [\texttt{m}][\texttt{a}]\tau)}{\texttt{f}~\texttt{a}~\texttt{b}}{e}
\]
is not (note the swapped dimensions in the type of \texttt{v}).

Another interesting case is for functions and operators, because of
the named parameters.  We assume that all arguments to the function
are variable names.  Intuitively, we then construct a substitution
$S_{p}$ from the parameter names in the type of the function to the
concrete names of the arguments, and apply this substitution before
checking whether the argument types match the parameter types.  The
substitution is also applied to the result type.

\begin{figure}[bt]
  \textrm{\textit{Further existentialising an existential type}}
  \begin{align*}
    \textrm{ext}(\nseq{x}{n},\exists\nseq{d}{m}.\rho) = %
    \exists \seq{y}~\nseq{d}{m}.\rho \\
    \textrm{where}~\seq{y} = \textrm{those of $\nseq{x}{n}$ free in $\rho$}
  \end{align*}

  \sempart{Expressions}{\Gamma \vd e : \exists\seq{d}.\rho}

  \fracc{
    \Gamma \vd e_1 : \exists\nseq{d}{l}.\rho
    \\
    p = (\nseq{d^p}{l},\nseq{y}{m})
    \\
    \exists S_r.(\textrm{$S_{r}$ is bijection from $\nseq{d^{p}}{l}$ to $\nseq{d}{l}$}) \wedge S_r(\rho) = (\nseq{\texttt{i32}}{l}, \nseq{\tau'}{m})
    \\
    \Gamma,p \vd e_2 : \exists\nseq{d^r}{k}.\rho'
  }{
    \Let{p}{e_1}{e_2} : \textrm{ext}(\nseq{x}{n}, \exists\nseq{d^r}{k}.\rho')
  }

\fracc{
  \textrm{lookup}_{\textrm{fun}}(f) = \nseq{(p_i: \tau_i)}{n} \rightarrow \exists\nseq{d}{l}\nseq{\tau'}{m}
    \\
    S = \langle \nseq{p_i \mapsto x_i}{n} \rangle \quad \forall i \in \Set{1,\ldots,n}.\Gamma(x_i) = S(\tau_i)
}
{\Gamma \vd f~x_{1}~\ldots~x_n : \exists\nseq{d}{l}.S(\nseq{\tau'}{m})}

\fracc{
  \Gamma \vda a_i : \etau^p_i \sp i \in \Set{1, \ldots, n}\\
  \textrm{TySch}(\emph{op}) = \nseq{(p_i: \etau_i)}{n} \rightarrow \exists\nseq{d}{l}.\nseq{\tau'}{m} \\
  S = \langle \nseq{p_i \mapsto x_i}{n}~\textrm{where $x_{i}=a_{i}$} \rangle \quad \forall i \in \Set{1,\ldots,n}.\etau^p_i = S(\etau_i)}{
  \Gamma \vd \emph{op}~\nseq{a}{n} : \exists\nseq{d}{l}.S(\nseq{\tau'}{m})
}

\fracc{
  \Gamma(s) = \texttt{bool} \\
  \Gamma \vd e_1 : \exists \seq{x}.\rho_1 \sp \Gamma \vd e_2 : \exists \seq{y}.\rho_2
  \\
  (\exists\seq{z}.\rho') = \textrm{Unification of $\seq{x}.\rho_1$ and $\seq{y}.\rho_2$}
}{
  \Gamma \vd \Let{p}{\If{s}{e_1}{e_2}}{e_3} : \exists\seq{z}.\rho'
}

  \caption{Size-aware typing rules for interesting expressions.}
  \label{fig:sizeTypeRules}
\end{figure}

\section{Size Inference by Example}
\label{subsec:size-analysis-intuition}

Thi section demonstrates, by example, the code transformation that (i)
makes explicit in the code the shape-dependent types and verifies the
assumed invariants and (ii) optimizes away in many cases the
existential types.  Our philosophy is to initially use existential
types liberally, with the expectation that inlining and simplification
will eventually remove almost all of them.

\begin{figure}
\begin{lstlisting}
let concat (xs: []f64) (ys: []f64): []f64 =
  let a = size 0 xs
  let b = size 0 ys
  let c = a+b
  let (is: []i32) = iota c
  let (zs: []f64) = map (\i -> if i < a then xs[i] else ys[i+b])
                        is
  in zs

let f (vss: [][]f64): []f64 =
  let (ys: []f64) (zs: []f64) =
    map (\(vs: []f64) ->
          let ys = reduce (+) 0.0 vs
          let zs = reduce (*) 1.0 vs
          in (ys, zs))
        vss
  let (rs: []f64) = concat ys zs
  in rs

let main (vsss: [][][]f64): [][]f64 =
  let (rss: [][]f64) =
    map (\(vss: [][]f64): []f64 ->
           let rs = f vss
           in rs) vs
  in rss
\end{lstlisting}

\caption{Running example: Program in un-sized IR.} 
\label{fig:RunEgSrc}
\end{figure}

The program in \Cref{fig:RunEgSrc} receives as input a
three-dimensional array \texttt{vsss}, and produces a two-dimensional
array, by mapping the elements of the outermost dimension of
\texttt{vsss} by function \texttt{f}.

\begin{figure}
\begin{lstlisting}
let concat @(n: i32) (m: i32)@ (xs: [@n@]f64) (ys: [@m@]f64): @d.@[@d@]f64 =
  let a = @n@
  let b = @m@
  let c = a+b
  let (is: [@c@]i32) = iota c
  let (zs: [@c@]f64) =
   map (\i -> if i < n then xs[i] else ys[i+a]) is
  in zs

let f @(m: i32) (k: i32)@ (vss: [@m@][@k@]f64): @d@.[@d@]f64 =
  let (ys: [@m@]f64) (zs: [@m@]f64) =
    map (\(vs: [@k@]f64) ->
          let ys = reduce (+) 0.0 vs
          let zs = reduce (*) 1.0 vs
          in (ys, zs))
        vss
  let @(l: i32)@ (rss: [@l@]f64) = concat @m m@ ys zs
  in rs

let main @(n: i32) (m: i32) (k: i32)@
         (vsss: [@n@][@m@][@k@]f64): @d1 d2.@[@d@][@d1@][@d2@]f64 =
@  let l = if n != 0@
@          then let (d: i32) (ws: [d]f64) = f n m vsss[0]@
@               in d@
@          else 0@
  let (rss: [@n@][@d@]f64) =
    map (\(vss: [@m@][@k@]f64): [@l@]f64 ->
           let @(d: i32)@ @(w: [d]f64)@ = f @m k@ vss
@           let vs = reshape l w@
           in vs)
        vsss
  in rss
\end{lstlisting}

  \caption{Running example:
    $\exists$-quantified target IR.  Changes compared to
    \Cref{fig:RunEgSrc} highlighted in red.}
\label{fig:RunEgTgt}
\end{figure}

The first stage, demonstrated in \Cref{fig:RunEgTgt}, transforms
the program into an unoptimised version in which (i) all arrays have
shape-dependent types, which may be existentially quantified, and (ii)
all ``assumed'' invariants are explicitly checked.
%
This is achieved by:
\begin{itemize}
\item Extending the function signatures to encompass also the shape
  information for each array argument.  For example, \lstinline{f}
  takes additional parameters \lstinline{m} and \lstinline{k} that
  specify the shape of array argument \lstinline{vss},

\item Representing function's array results via
  existentially-quantified shape-dependent types.  For example, the
  return type of \lstinline{f} is specified as
  \lstinline{d.[d]f64}, indicating an existential size \lstinline{d}.

\item Modifying \lstinline{let} patterns to also explicitly bind any
  existential sizes returned by the corresponding expression.  For
  example, the binding of the result of \lstinline{concat} now also
  includes a variable \lstinline{l}, representing the result size.

\item For the \lstinline{map} in \lstinline{main}, we need to make a
  ``guess'' at the size of the array being returned, which we store as
  the variable \lstinline{l}.  This guess is made by applying the
  \kw{map} function to \lstinline{vsss[0]}, which produces both a size
  and an array, from which we use just the size.  This corresponds to
  \textit{slicing} the \kw{map} function.  If \lstinline{vsss} is
  empty (that is, if \lstinline{n} is zero), the guess is zero.

  Since \lstinline{f} returns an existential result, and the lambda
  \textit{must} return an array of exactly type \lstinline{[l]f64}, we
  use a \lstinline{reshape} to obtain this desired type.  Since the
  \lstinline{reshape} fails if \lstinline{d != n}, this effectively
  ensures that the \lstinline{map} produces a regular array.

\item Since all arrays in scope also have variables in scope for
  describing their size, replace all uses of \lstinline{size} with
  references to those varuables.
\end{itemize}

It is important to note that this transformation preserves
asymptotically the number of operations of the original program.
However, it performs a significant amount of redundant computation.
To compute \lstinline{l}, we compute the entire result, only to throw
most of it away.  General-purpose optimisation techniques can be
employed to eliminate the overhead.  On
\Cref{fig:SimplifyFShape} we see the result of inlining all
functions, followed by straightforward simplification, dead code
removal, and hoisting of the computation of \lstinline{c}.  The result
is that all arrays constructed inside the \lstinline{map}s have a size
that can be computed before the \lstinline{map}s are entered.  From an
operational perspective, this lets us pre-allocate memory before
executing the \lstinline{map}s on a GPU.  This is essential for GPU
execution because dynamic allocation and assertions are typically not
well suited for accelerators, hence the shapes of the result and of
various intermediate arrays need to be computed (or at least
overestimated) and verified before the kernel is run.

\begin{figure}
\begin{lstlisting}
let main (n: i32) (m: i32) (k: i32)
         (vsss: [n][m][k]f64): d1 d2.[d][d1][d2]f64 =
  let l = if n != 0 then m+m else 0
  let c = m+m
  let (rss: [n][d]f64) =
    map (\(vss: [m][k]f64): [l]f64 ->
          let (ys: [m]f64) (zs: [m]f64) =
            map (\(vs: [k]f64) ->
                  let ys = reduce (+) 0.0 vs
                  let zs = reduce (*) 1.0 vs
                  in (ys, zs))
                vss
          let (is: [c]i32) = iota c
          let (zs: [c]f64) =
            map (\i -> if i < n then xs[i] else ys[i+a]) is
          let rs = reshape l zs
          in rs)
        vsss
  in rss
\end{lstlisting}

  \caption{After inlining all functions and performing simple
    inlining, dead-code elimination, and simplification---no
    existential quantification left.}
\label{fig:SimplifyFShape}
\end{figure}

However, there is still a problem with the current form of the code.
The issue is that the compiler cannot statically see that
\lstinline{l==m}, and thus has to maintain the \lstinline{reshape} and
perform a dynamic safety check at run-time.  This is because the
computation of \lstinline{l} is hidden behind a branch.  The branch
was conservatively inserted because we could not be sure that the
value of \lstinline{vsss[0]} would not be used for computing the size
(size analysis is intraprocedural, and so we have no insight in the
definition of \lstinline{f}), but now it is a hindrance to further
simplification.  There are at least two possible solutions, both of
which are used by the present Futhark compiler:

\begin{enumerate}
\item Give the programmer the ability to annotate the original lambda
  (in the source language) with the return type, \textit{including}
  expected size.  This effectively lets the programmer make the guess
  for us, with no branch required.  The result is still checked by a
  \lstinline{reshape}, but in most cases the guess will be statically
  the same as the computed size, and the \lstinline{reshape} can thus
  be simplified away.
\item Somehow ``mark'' the branch as being a size computation.  Then,
  after inlining and simplification, we can recognise such branches,
  and simplify them to their ``true'' branch, if that branch contains
  only ``safe'' expressions, where a safe expression is one whose
  evaluation can never fail.  We have to wait until after inlining and
  simplification, as a function call can never be considered safe.

  This solution has the downside that it may affect whether an inner
  size of an empty array is zero or nonzero.
\end{enumerate}

In practise, the first solution is preferable in the vast majority of
cases, as it also serves as useful documentation of programmer intent
in the source program.

A third solution is to factor out the ``checking'' part of the
\lstinline{reshape} operation.  This approach is sketched on
\Cref{fig:SimplifyFShapeCert}.  Here, we use a hypothetical
\lstinline{assert} expression for computing a ``certificate'', on
which the \lstinline{reshape} expression itself is predicated.  To
enable the check to be hoisted safely out of the outer
\lstinline{map}, the condition also succeeds if the outer map contains
no iterations (\lstinline{n == 0}).  The Futhark compiler currently
makes only limited use of this technique, as the static equivalence of
sizes is a more powerful enabler of other optimisations.

One may observe that in the resulting code, the shape and regularity
of \texttt{rss} are computed and verified before the definition of
\texttt{rss}, respectively, and, most importantly, that the size
computation and assumed-invariant verification introduce negligible
overhead, i.e., $O(1)$ number of operations.

\begin{figure}
\begin{lstlisting}
let main (n: i32) (m: i32) (k: i32)
         (vsss: [n][m][k]f64): d1 d2.[d][d1][d2]f64 =
  let l = if n != 0 then m+m else 0
  let c = m+m
@  let cert = assert(n == 0 || l==c)@
  let (rss: [n][d]f64) =
    map (\(vss: [m][k]f64): [l]f64 ->
          let (ys: [m]f64) (zs: [m]f64) =
            map (\(vs: [k]f64) ->
                  let ys = reduce (+) 0.0 vs
                  let zs = reduce (*) 1.0 vs
                  in (ys, zs))
                vss
          let (is: [c]i32) = iota c
          let (zs: [c]f64) =
            map (\i -> if i < n then xs[i] else ys[i+a]) is
          let rs = reshape@<cert>@ l zs
          in rs)
        vsss
  in rss
\end{lstlisting}

  \caption{Separating the size-checking of a \lstinline{reshape} from
    the \lstinline{reshape} itself.}
\label{fig:SimplifyFShapeCert}
\end{figure}

Note that the code shown on \Cref{fig:RunEgTgt} is the only
\textit{required} step we have to perform.  Subsequent optimisation to
eliminate existential quantification could be done in any way that is
found desirable, perhaps via interprocedural analysis or slicing.
Previously, we experienced with a technique based on \textit{slicing},
where a function \lstinline{g} is divided into two functions
\lstinline{g_shape} and \lstinline{g_value}, the first of which
computes the sizes of all (top-level) arrays in the latter, including
the result.  An example is shown on \Cref{fig:FShapeSlice},
which contains a portion of the running example.  The
\lstinline{concat} function has been split into
\lstinline{concat_shape} and \lstinline{concat_value}. The call to
\lstinline{concat} has been likewise split.  The Futhark compiler
currently does not use this approach.  Partly because of the risk for
asymptotic slowdown in the presence of recursion (which was supported
at the time), and partly because merely inlining plus simplification
is easier to implement, and performed equally well.

\begin{figure}

\begin{lstlisting}
let concat_shape (n: i32) (m: i32) (xs: [n]f64) (ys: [m]f64): i32 =
  n+m

let concat_value (n: i32) (m: i32) (c: i32) (xs: [n]f64) (ys: [m]f64): [c]f64 =
  let (is: [c]i32) = iota c
  let (zs: [c]f64) =
    map (\i -> if i < n then xs[i] else ys[i+n]) is
  in zs

let f (m: i32) (k: i32) (vss: [m][k]f64): d.[d]f64 =
  let (ys: [m]f64) (zs: [m]f64) =
    map (\(vs: [k]f64) ->
          let ys = reduce (+) 0.0 vs
          let zs = reduce (*) 1.0 vs
          in (ys, zs))
        vss
  let (l: i32) = concat_shape m m c ys zs
  let (rss: [l]f64) = concat_value m m c ys zs
  in rs
\end{lstlisting}

  \caption{An example of applying the slicing approach to \lstinline{concat}.}
  \label{fig:FShapeSlice}
\end{figure}

\section{Related Work}

An important piece of related work is the work on the FISh \cite{fish}
programming language, which uses partial evaluation and program
specialization for resolving shape information at compile time.  The
semantics of FISh guarantees that this is possible.  Futhark uses
approximately the same strategy, but extends it to handle constructs
such as \kw{filter}, at the cost of no longer being able to fully
resolve shape informatation statically in all cases.

Much work has also gone into investigating expressive type systems,
based on dependent types, which allow for expressing more accurately,
the assumptions of higher-order operators for array operations
\cite{AgdaAccellerate,Trojahner:2008,Trojahner2009643}. Compared to
the present work, such type systems may give the programmer certainty
about particular execution strategies implemented by a backend
compiler. The expressiveness, however, comes at a price. Advanced
dependent type systems are often very difficult to program and
modularity and reusability of library routines require the end
programmer to grasp, in detail, the underlying, often complicated,
type system.  Computer algebra systems~\cite{AXIOM,AldorCAH} have also
provided for a long time a compelling application of dependent types
in order to express accurately the rich mathematical structure of
applications, but inter-operating across such systems remains a
significant challenge~\cite{mapal_synasc,alma:ISSAC}.

A somewhat orthogonal approach has been to extend the language
operators such that size and bounds checking invariants always
hold~\cite{ElsmanDybdal:Array:2014}, the downfall being that
non-affine indexing might appear.  The Futhark strategy is instead to
rely on advanced program analysis and compilation techniques to
implement a pay-as-you-go scheme for programming massively parallel
architectures.

Another strand of related work is the work on effect systems for
region-based memory management \cite{mlkit_retrospective}, in
particular, the work on multiplicity inference and region
representation analysis in terms of physical-size inference
\cite{vejlstrup94,btv96}. Whereas the goal of multiplicity inference
is to determine an upper bound to the number of objects stored into a
region at runtime, physical-size inference seeks to compute an upper
bound to the number of bytes stored in a single write. Compared to the
present work, multiplicity inference and physical-size inference are
engineered to work well for common objects such as pairs and closures,
but the techniques work less well with objects whose sizes are
determined dynamically.

%%% Local Variables:
%%% mode: latex
%%% TeX-master: "thesis"
%%% End:
